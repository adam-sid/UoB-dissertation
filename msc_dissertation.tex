\documentclass[11pt]{article}

%TC:ignore
% Packages
\usepackage[english]{babel}
\usepackage[T1]{fontenc}
\usepackage{graphicx}
\usepackage[a4paper, margin=2.5cm]{geometry}
\usepackage{lipsum} % Lorem ipsum text
\usepackage{etoolbox} % For patching commands
\usepackage{titlesec} % Section/part formatting
\usepackage{fancyhdr} % Custom headers/footers
\usepackage{tocloft}  % Table of contents settings
\usepackage[style=ieee]{biblatex} % Citations
\usepackage{csquotes}
\usepackage{siunitx} % SI units
\usepackage{parskip}

% Global variables
\newcommand{\myName}{Adam Sidnell}
\newcommand{\mySupervisor}{Professor Ruzanna Chitchyan}
\newcommand{\mySubmissionMonthYear}{September 2025}
\newcommand{\myReportTitle}{[TBD: NAME OF TOOL]}
\newcommand{\myReportSubtitle}{TBD: A tool to study microclimates in an orchard}
\newcommand{\wordCount}{TBD}
\newcommand{\farmName}{Small Brook Farm}
\newcommand{\farmLocation}{Devon}

% Section formatting
\titleformat{\section}{\normalfont\LARGE\bfseries}{\thesection}{1em}{}
\titlespacing*{\section}{0pt}{0pt}{0.5cm}

% Header/footer setup
\pagestyle{fancy}
\fancyhf{}
\fancyhead[R]{\leftmark}
\fancyfoot[C]{\thepage}
\renewcommand{\headrulewidth}{0.4pt}
\renewcommand{\sectionmark}[1]{\markboth{\thesection\ \MakeUppercase{#1}}{}}

% Custom section-start page style (no header)
\fancypagestyle{sectionstart}{
  \fancyhf{}
  \fancyfoot[C]{\thepage}
  \renewcommand{\headrulewidth}{0pt}
}

% Clear page & vertical space before sections
\makeatletter
\pretocmd{\section}{\clearpage\vspace*{3cm}\thispagestyle{sectionstart}}{}{}
\makeatother

% Part page formatting (centred on page, "part" smaller than "name")
\titleformat{\part}[display]
{\normalfont\centering\bfseries}
{\vfill \LARGE \partname\ \thepart}
{1em}
{\Huge }
[
  \vfill
  \thispagestyle{partpage}
  \clearpage
]

% Custom part page style
\fancypagestyle{partpage}{
  \fancyhf{}
  \fancyfoot[C]{\thepage}
  \renewcommand{\headrulewidth}{0pt}
}

% Dotted lines in contents
\renewcommand{\cftsecleader}{\cftdotfill{\cftdotsep}}

% Bibliography file
\addbibresource{references.bib}

% Contents title
\addto\captionsenglish{\renewcommand{\contentsname}{Table of Contents}}



\begin{document}

% Roman page numbers for preamble
\pagenumbering{roman}

% Title page
\begin{titlepage}
    \centering

    \vspace*{2.2cm}

    % Top bar
    \rule{\textwidth}{2pt} \\[-2ex]
    \rule{\textwidth}{0.5pt} \\[0.7cm]

    % Title and description
    {\LARGE\bfseries \myReportTitle} \\[0.3cm]

    \parbox{0.89\textwidth}{\centering
        {\large \textit{\myReportSubtitle}}
    } \\[0.3cm]

    % Bottom bar
    \rule{\textwidth}{0.5pt} \\[-2ex]
    \rule{\textwidth}{2pt} \\[0.3cm]

    % Author and supervisor
    {\large By} \\[0.3cm]
    {\large \myName} \\[0.3cm]
    {\normalsize Supervised by \mySupervisor} \\[1cm]

    % Logo 
    \includegraphics[width=0.2\textwidth]{contents/title/fig0/uob_logo.png} \\[1cm]

    % Department and Date
    \begin{minipage}{0.65\textwidth}
        \centering
        {\Large Department of Computer Science} \\[0.3cm]
        {\Large \textsc{UNIVERSITY OF BRISTOL}} \\[1cm]
        {\normalsize A dissertation submitted to the University of Bristol
        in accordance with the requirements of the degree of \textsc{Master of Science}
        in the Faculty of Engineering} \\[1cm]
        {\large \mySubmissionMonthYear}
    \end{minipage}

    \vspace*{1cm}


    \begin{flushright}
        Word count: \wordCount
    \end{flushright}

    \vspace*{2.2cm}

\end{titlepage}


% Executive summary
\section*{Executive summary}
This project report presents the design, development and deployment of a
complete IoT weather station network for use in agricultural settings. The
network was built from the ground up with off-the-shelf electronics and a custom
weather proofed enclosure. The system streams one minute sensor measurements to
a linked web application also developed in this project called Agriscanner
(\url{https://agriscanner.onrender.com}).

The Agriscanner webapp provides users with historical, real-time, and forecast
weather data through an intuitive and accessible interface. The webapp uses a
machine learning model (LightGBM), to display microclimate specific forecast
data that is trained by comparing general forecast information to sensor
readings from the IoT network.

General forecasting models often miss smaller climate variations that exist in
farms and fields. While commercial weather monitoring stations are often a
trade-off between the range they can transmit information and their price.
Because farms are remote, any device collecting information must be
self-sustaining in power and able to send data over long distances.

The weather stations in this design achieved a measured range of 1,200m from the
receiver in non-ideal circumstances and the final design is estimated to have
over double the range of commercial options due to the inclusion of a repeater.

On the software side, the backend was built with a relational database that
keeps a log of all sensor readings, along with a secure API that facilitates
these database queries. It serves a front-end web interface that scored highly
for usability on both mobile and desktop.

The machine learning model's forecast was compared to actual sensor data and
showed that it was more accurate than the general forecast in most scenarios,
but more training data is needed to improve its accuracy further.

It was an ambitious project for the timescale - involving the design and
construction of four hardware devices; the development of a backend database and
API; the development of frontend web interface and the training and deployment
of a set of machine learning models.

\addcontentsline{toc}{section}{Executive summary}


% Dedication and acknowledgments
\newpage
\section*{Dedication and acknowledgments}
I would like to thank my parents for the use of their garden, their wisdom and
their near infinite patience; my supervisor for her advice and support
throughout this project; and my partner, for helping me with range testing the
network by walking the length of the Downs with a box of electronics, despite
strange looks from passers-by.


\addcontentsline{toc}{section}{Dedication and acknowledgments}


% Declaration
\newpage
\section*{Author's declaration}
I declare that the work in this dissertation was carried out in accordance with
the requirements of the University's Regulations and Code of Practice for
Research Degree Programmes and that it has not been submitted for any other
academic award. Except where indicated by specific reference in the text, the
work is the candidate's own work. Work done in collaboration with, or with the
assistance of, others, is indicated as such. Any views expressed in the
dissertation are those of the author.


SIGNED:\qquad Adam Sidnell \qquad \qquad \qquad DATE:\qquad    2nd September
2025
\addcontentsline{toc}{section}{Author's declaration}


% Contents
\clearpage
\begingroup
\pagestyle{plain}
\vspace*{3cm}
\tableofcontents
\clearpage
\endgroup

% Figures
\clearpage
\begingroup
\pagestyle{plain}
\vspace*{3cm}
\listoffigures
\clearpage
\endgroup
%TC:endignore

% Use arabic numbers for main contents pages
\pagenumbering{arabic}

% Introduction
\section{Introduction}

In this report, I describe an online tool called \myReportTitle{}, which
collects and displays microclimate data from an apple orchard via an IoT weather
sensor network that I designed and built. The tool aims to enable microclimate
specific forecasting and provide farmers with accessible, location specific
climate data to support decision making and enable microclimate specific
forecasting. The IoT hardware was built using off-the-shelf components and
housed in hand made enclosures. The deployed system consisted of four
components: two field sensor nodes capable of reading weather data and
transmitting it using a modern radio protocol called LoRa; a repeater that
boosts received LoRa signals; and an internet connected gateway that uploads the
weather data to an online database. The sensor nodes were placed at two
locations on the farm that the owner identified as having distinct microclimatic
conditions. This allowed for the collection of weather data tailored to specific
areas of interest, which was then displayed on the Agriscanner web application.


\subsection{Aims and contributions}

The main aim of this project was to develop a low-cost, low-maintenance IoT
solution that helps farmers monitor microclimate conditions and make better
informed decisions. The final hardware system needed to be physically robust to
survive outside while the the software needed to present relevant information in
an intuitive way, ensuring that users can easily make use of the data.

Weather is of critical importance to farmers, as it is an important factor in
final yield and helps to predict the optimal time to harvest, the likelihood of
disease growth and irrigation needs. However, weather forecasts are typically
based on data from distant weather stations and large scale models which fail to
capture variations that exist within a single farm. These local variations,
known as microclimates, can differ significantly even across relatively short
distances due to differences in elevation, tree cover, or soil conditions.

The key contributions of this paper are:

\begin{itemize}
    \item The development of a low cost, ultra long-range remote weather station
    system with superior range and value to current solutions on the market.
    \item A publicly available web application that visualises live data from
    multiple field locations.
    \item A method for enabling microclimate forecasting by comparing local
    sensor data with broader regional weather forecasts.
\end{itemize}

\subsection{Layout of dissertation}

This dissertation has four parts. Part 1 provides the technical background to
understand the project, including an overview of the core technologies used and
a review of related work in the field. Part 2 details the hardware development
process, covering the rationale behind component selection, the testing of
hardware, and how the IoT nodes were deployed in the field. Part 3 focuses on
the software aspect of the project, outlining the design and implementation of
both the backend infrastructure and the frontend web interface. Part 4 then
offers a critical evaluation of the system, discussing its performance,
usability, and limitations, as well as highlighting opportunities for future
development improvements and research.



% Background
\part{Background}

% Internet of things

\section{LoRa}\label{sec:lora}

Before analysing IoT systems more generally, I will explain the most critical
hardware enabling technology for this project, which is the radio communication
technique known as LoRa.

\subsection{What is LoRa?}

LoRa stands for \textbf{Lo}ng \textbf{Ra}nge and it is a radio modulation
technique invented in 2014 that allows for the transmission of data over very
long distances. It is one of several competing low power wide area networks
(LPWAN) but the hardware to use it is much more available than these other
networks. LoRa has a range over 4000 times greater than WiFi \cite{spiess2019},
a range that it can achieve with remarkably little power (Table
\ref{tab:lora-stats}). This makes LoRa the preferred technology choice for the
remote, off-grid application in this project.

For a brief overview of the principles behind LoRa I have written the
supplementary section "How LoRa works" in Appendix \ref{app:lora-explained}.
However, to summarise this, the most important principle that separates LoRa
from traditional radio modulation is the fact that it can demodulate incoming
signals more efficiently.

\subsection{Benefits and limitations}\label{sec:lora-benefits}

The benefit from the ability of the LoRa receiver to demodulate signals more
efficiently is two-fold. First it reduces the power needs on transmitter and
receiver: the transmitter sends less powerful signals because the receiver can
more easily distinguish signals; in turn the receiver can demodulate with a
lower power budget because of the easier correlation process with LoRa chirps
(Figure \ref{fig:lora-wave} from supplementary Appendix).

The second related benefit is the ability for the receiver to demodulate signals
which are below the noise floor (the sum of all interfering signals).
Essentially, even when background noise is 'louder' than the LoRa signal, the
receiver is still able to distinguish and process the data in the signal. This
helps LoRa transmitters to broadcast signals with a far greater effective range
despite its low power. Table \ref{tab:lora-stats} summarises LoRa against other
well known wireless communication technologies, showing the relative advantages
or LoRa. The main disadvantage of LoRa is that it has a comparatively lower data
throughput, however the packets involved in this project were only around 20-30
bytes so this was not a concern.

\begin{table}[ht]
  \centering
  \begin{tabular}{|l|l|p{4.5cm}|r|}
    \hline
    \textbf{Technology} & \textbf{Wireless Communication} & \textbf{Range} &
    \textbf{Tx Power} \\
    \hline
    Bluetooth           & Short range                     & 10 m & 2.5 mW  \\
    \hline
    WiFi                & Short range                     & 50 m & 80 mW   \\
    \hline
    3G/4G               & Mobile network                  & 5,000m & 5000 mW \\
    \hline
    LoRa                & LPWAN                           &
    \parbox[t]{4.5cm}{2,000--15,000m} & 20 mW   \\
    \hline
  \end{tabular}
  \caption{Comparison of wireless technologies (Source:
    \cite{lie_lora_readthedocs})}
  \label{tab:lora-stats}
\end{table}


\section{\emph{Internet of Things}}

\subsection{What is IoT?}

The Internet of Things (IoT) refers to the concept of integrating networking
capability in a range of devices, allowing for cooperation to reach common goals
\cite{atzori2010}. A similar but more colloquial term would be "smart" devices.

The number of IoT devices is forecasted to reach 40 billion by 2030 from 16.6
billion in late 2023 \cite{sinha2024}. IoT devices are already used extensively
in a multitude of commercial and domestic settings.

\subsection{The layers of IoT}

All IoT relies on the existence of three fundamental network layers
\cite{burhan2018iot}, working from the bottom to the top there is:

\begin{enumerate}
  \item Perception - This layer contains sensors that collect information about
  conditions in the world around them. This layer may perform some
  transformations on the data it receives (e.g. sorting, formatting) or just
  transmit raw data. An example of this is a smart car charger collecting data
  on the charge level of an electric car.
  \item Network - This layer acts as the bridge between the perception layer and
  the application layer. It is the medium and protocols associated with the
  transmission of data and the hardware necessary to interpret this. Following
  the above example this could be the use of WiFi or a mesh protocol such as
  Zigby to transmit the car's current charge level to a gateway - such as a WiFi
  router.
  \item Application - This encompasses any software that manipulates, displays
  or otherwise uses data from the perception layer. This could be hosted on the
  cloud or locally. In the example above this could be an app that shows the
  current charge status of the car.
\end{enumerate}

\subsection{IoT enabling technologies}

The surge in IoT for the past few decades has been fuelled by the emergence of
new technologies. Here I will explore the technologies that have are most
relevant to the weather stations built for this project.

\begin{enumerate}
  \item Efficiency improvements in microchips - breakthroughs in microchip
  fabrication have led to smaller more efficient chips with improved
  performance.
  \item Lithium-Ion batteries improvements - continuous improvements in the
  energy density of lithium-ion batteries has made it possible to power devices
  for long periods without mains power.
  \item Low-power long-range radio - new radio communication techniques such as
  LoRa allow for data transmission over several kilometres using a fraction of
  the power required by traditional mobile or Wi-Fi technologies.
  \item Affordability of solar panels - since 1970 the price of solar panels has
  decreased to 1/500th of its original cost \cite{economist2024} making solar a
  viable power source for IoT systems.
  \item Growth of hobbyist embedded systems - since the release of accessible
  platforms such as Arduino in 2005, the growth of hobby level embedded systems
  has lowered the barrier to entry to create IoT systems.
  \item Accessible cloud computing and hosting - Cheap and available web hosting
  has allowed application level systems to be more easily developed.
\end{enumerate}

\subsection{Studies using LoRa IoT weather stations in agriculture}

The use of IoT in agriculture has become widespread in recent years as it offers
the opportunity for farmers to improve yields and cut costs by bringing digital
solutions that would not have previously been viable without access to power and
internet connectivity. The use of IoT in agriculture is often wrapped up in the
moniker of "Smart Farming", which encompasses a range of digital, robotic, and
internet-enabled approaches to improving efficiency.

A 2019 review by Farooq et al. \cite{farooq2019iot} reveals the scope of IoT
applications in agriculture. These include precision farming (IoT weather
monitoring), automated irrigation, pest and disease prediction with machine
learning, and even the deployment of agricultural drones for spraying, mapping
and imaging. 

IoT weather stations like those developed in this project are a growing trend in
agriculture.

\subsection{Commercial IoT agricultural solutions with LoRa}

While IoT weather station solutions are available commercially there are issues
with using these. 

Virtually all affordable weather stations rely on WiFi to stay connected which
is not viable for most farms as there is a large separation between the fields
they need to be deployed in and buildings with WiFi access. As discussed in
section~\ref{sec:lora-benefits}, WiFi has just a fraction of the range of LoRa
so these types of stations are not comparable to my own.

For a detailed comparison of different models compared to my own see the
relevant section in the evaluation.



% Microclimates
\section{Microclimates}

This section gives some academic background on microclimates with a focus on
their relevance to agriculture, and then looks at related attempts to predict
microclimate weather.

\subsection{What is a microclimate?}

A microclimate is generally understood as a set of distinct climatic conditions
within a small, localised area \cite{MetOffice2023}. The maximum size of a
microclimate is debated, but the World Meteorological Organisation (WMO) regards
it as occupying an area of anywhere from less than one metre across to several
hundred metres \cite{wmo2024}.  In practice, microclimates can occur in spaces
such as gardens, valleys, caves, or fields. Even human-made structures can
generate their own microclimates; for example, tall buildings can create
\emph{street valleys} that reduce wind flow and lead to the formation of
localised pockets of warmer air, which  can trap higher concentrations of
pollution from vehicle emissions \cite{yang2023}. Vegetation plays a critical
role in influencing microclimates. The addition of trees to an urban environment
can reduce air temperature by as much as \SI{2.8}{\degreeCelsius}
\cite{lai2019}.

\subsection{Microclimates in agriculture}

Microclimatic variations have profound implications for agriculture, as the
climate that crops are exposed to has an enormous impact on overall agricultural
yields. Indeed, farmers have modified the microclimate of crop fields for
millennia, a clear example of this being the use of fencing to reduce soil
erosion and damage to edible plants \cite{cleugh1998}. Therefore, the
relationship between microclimates and agriculture has been the subject of
extensive research, particularly as climate change introduces new threats to
food security.

A Danish study by Haider et al. investigated how agricultural pests and diseases
are influenced by microclimatic conditions. Temperature sensors were installed
in six different areas with distinct microclimates (such as hedges and cattle
fields). The data revealed that the daytime temperature in these microclimates
was significantly higher than those predicted by Danish weather forecasts. Using
these measurements, the researchers then estimated the incubation periods of
various pests and diseases, demonstrating that elevated temperatures in
microclimates could shorten incubation times and thus accelerate the risk of
disease outbreaks \cite{haider2017}.

Another important aspect of microclimates for farmers is how it can affect frost
risk - sudden and unpredictable drops below freezing in crop fields that damage
plants and are particularly common in spring. A principal factor for this is a
lack of "cold-air drainage". In areas with depressed topography, cold, dense air
can accumulate to form pockets where temperatures can be several degrees lower
than the surrounding landscape \cite{drepper2022}. These local conditions are
not captured by general weather forecasts highlighting the need for more precise
monitoring and development of effective early warning systems.

\subsection{Microclimate prediction using machine learning}

There have been a number of recent studies where machine learning has been used
to help predict micro climate conditions in agricultural settings. General
weather models operate at magnitudes between 1 and 10 km and microclimate
predictions require models that operate at scales of roughly 100m or less. Using
current numerical weather prediction models for micro-scale predictions is
computationally expensive \cite{blunn2024machine}, and these models have lower
accuracy rates than predictions using machine learning due to the inherent
complexity and non-linear nature of microclimates.

A number of the studies described here focus on comparing the accuracy achieved
by different types of machine learning, and highlighted a distinction between
machine learning (ML) and deep learning (DL).  They are both systems that learn
from data and make decisions based on datasets provided when the systems are
trained.  ML techniques include decision trees, regression and neural networks.
DL systems are a type of ML that uses neural networks with many layers and is
used to model more complex patterns with larger datasets. The table below
provides a summary of the studies included in this section


\begin{table}[ht]
      \centering
      \scriptsize
      \setlength{\tabcolsep}{6pt}
      \renewcommand{\arraystretch}{0.98}
      \caption{Selected summary of machine learning studies}
      \label{tab:forecast-studies}
      \resizebox{\textwidth}{!}{%
            \begin{tabular}{|>{\centering\arraybackslash}m{2.2cm}
                  |>{\centering\arraybackslash}m{3.2cm}
                  |>{\centering\arraybackslash}m{1.8cm}
                  |>{\centering\arraybackslash}m{3.2cm}
                  |>{\centering\arraybackslash}m{3.0cm}
                  |>{\centering\arraybackslash}m{3.0cm}|}
                  \hline
                  \textbf{Study}                                    &
                  \textbf{Location}                         & \textbf{Type of
                  learning}   & \textbf{Learning framework} & \textbf{Source of
                  forecast data}          & \textbf{Source of training target
                  data} \\
                  \hline
                  Agriscanner (This work)                           & Outside
                  rural - South Gloucestershire, UK & ML & LightGBM &
                  OpenWeather                               & Sensor readings \\
                  \hline
                  Kumar et al.\ (2021)\cite{kumar2021}              & Various &
                                                                    DL & LSTM &
                                                                    Not
                                                                    specified &
                                                                    Sensor
                                                                    readings \\
                  \hline
                  Zanchi et al.\ (2023)\cite{zanchi2023harnessing}  & Outside
                  rural; Italy, Lombardy & DL & Feed-forward neural network &
                  \makecell[c]{ERA5 \\ ARPA } & Sensor readings \\
                  \hline
                  Blunn et al.\ (2024)\cite{blunn2024machine}       & Outside
                  urban; London, UK                 & ML \& DL &
                  \makecell[c]{Multilayer Perceptron \\ Random Forest \\
                  XGBoost}                                          & UKV
                  weather data                          & Citizen weather
                  stations                                        \\
                  \hline
                  Abdelmadjid (2025)\cite{abdelmadjid2025enhancing} & Greenhouse
                  - Unknown                      & ML \& DL &
                  \makecell[c]{CNN-LSTM \\ LSTM (DL) \\ SVM-RBF \\ Prophet \\
                  LightGBM \\
                  XGBoost}                                          & Public
                  kaggle dataset                     & Public kaggle dataset \\
                  \hline
            \end{tabular}%
      }
\end{table}

A 2021 study by Kumar et al \cite{kumar2021} developed an ML framework called
DeepMC as a part of a Microsoft Research initiative. Their model is able to
predict a variety of climatic variables such as soil moisture, wind speed and
temperature using inputs from weather station forecasts and IoT sensors. They
were able to get up to 90\% accuracy with a 12-120 hour forecast range.

Zanchi et al \cite{zanchi2023harnessing} used physical modelling of local
terrain combined with DL to forecast the microclimate in the foothills of
Lombardy. The objective was to predict the local conditions at the meter-scale
as opposed to the 10km+ scale of regional and global weather forecasts. The
initial model combined data about the morphology of the local terrain and
weather forecast data to provide the input data for two feed-forward neural
networks. These neural networks were trained to predict the local weather
variables using data from 25 sensors deployed in the region being studied.  The
study demonstrated that local predictions were more accurate when using forecast
data from local weather stations (ARPA) as opposed to global climate datasets
(ERA5), but accuracy was still good either.

This study is of particular interest for this dissertation as it used IoT
sensors to generate the training targets for the neural networks and in this
regard is the most similar to my design. The paper also includes a discussion of
problems with reliability of the sensors - only 4 out of 25 units ran without
failure over the period of the study - and the need to clean the data to ensure
these failures were excluded from the training data. In my project, I came to
appreciate how time consuming the development of hardware is, and how many
unexpected issues can occur. As this project had a hard end date, this affected
the quantity of data I was able to collect and therefore ultimately impacted the
training of the machine learning models.

Blunn et al \cite{blunn2024machine} ran a study focussed on predicting
temperatures in urban environments during heatwaves, using data from eight
heatwaves in London, UK. They used data from the UKV - a high-resolution weather
forecasting model - and from citizen weather stations (CWS), made available via
the Weather Observations Website - a cloud platform where individuals can upload
data from their personal weather stations. The study tested a variety of ML
models including one DL model. The authors used a similar model training design
to that used in this dissertation. The models were trained on UKV variables
(i.e. a general forecast as with mine) and CWS variables (i.e. sensor
information) to bias correct the UKV readings and create a forecast prediction
model that could predict the CWS readings accurately (mean average error:
0.12\(^\circ\)C) compared with the general weather readings from UKV (mean
average error: 0.64\(^\circ\)C). The main point of difference to my own study
here is that I am using four different climate variables compared to just
temperature in the Blunn paper. Also CWS is publically available civilian
weather stations whereas I am using my own sensor data from a devices I have
built.

A very recent paper from Abdelmadjid et al 2025 \cite{abdelmadjid2025enhancing}
used online datasets from Kaggle to develop an ML tool to predict changes in
temperature and humidity within greenhouses in response to changes to external
weather conditions.  They used this data to test three ML models and three DL
models and selected the LightGBM ML model and the LSTM DL model as the best
performing models for prediction. The overall system design consisted of four
LSTM models feeding into the LightGBM model. This design resulted in 98.45\%
accuracy for temperature predictions and 99.61\% accuracy for humidity
predictions. This study informed my decision to use the LightGBM model for my
system as they found it  to be reliable and effective in capturing underlying
data patterns. 

\part{Hardware development}

\section{Overview of hardware}\label{sec:hardware-overview}

The design of the IoT system involved three distinct module types: the sensor
nodes and repeater which are located outside and the gateway which is located
inside a building with WiFi access. The system consists of two separate sensor
nodes that are placed in range of the repeater. The repeater is then placed in
range of the gateway endpoint to allow the uploading of climate data to the
cloud.

\begin{figure}[H]
      \centering
      \includegraphics[width=0.9\textwidth]{contents/part-2/fig2/network-diagram.png}
      \caption{Network diagram of the system}
      \label{fig:network-diagram}
\end{figure} 

\begin{enumerate}
      \item Sensor Node: The two nodes in this part of the network are in the
            perception layer. The nodes collect readings on temperature, humidity,
            wind speed and soil moisture levels. The results are collated into a
            comma separated string which is then emitted as a single packet from
            the LoRa transmitter.
      \item Repeater: This module is part of the network layer of the system as it
            facilitates communication between the perception layer and the
            application layer. The addition of a repeater node effectively doubles
            the range of the system. The repeater reads and decodes received
            LoRa signals from the sensor nodes. It then adds signal strength
            information to the string and re-emits the LoRa signal. The repeater
            has no sensors but is otherwise identical to the sensor nodes.
      \item Gateway: The final part of the hardware system is the gateway - which
            is also part of the network layer. This module has mains power and a
            WiFi connection. The gateway receives LoRa signals from the receiver
            and uploads the data to the cloud.
\end{enumerate}





\section{Design}

\subsection{Node Components}

As the sensor nodes and the repeater are situated outside and rely on solar
power, they required components that were highly power-efficient, while also
being capable of transmitting small data packets via LoRa. An equally important
consideration for the design was how to weatherproof the final enclosure to
protect the sensitive electronics from water ingress and environmental damage.

\subsubsection{Challenger RP2040 LoRa}

The iLabs Challenger RP2040 LoRa (Figure~\ref{fig:challenger-rp2040} )is an
embedded computer that uses the Raspberry Pi RP2040 chip that was released in
2021. The RP2040 itself is a low-cost and power-efficient processor easily
capable of performing the data encoding and transmission in my use case.
Additionally, the chip is extremely popular with over 10 million units being
produced in the first two years of release \cite{pounder2023}. This popularity
means there is ample documentation for developing with this processor.

\begin{figure}[H]
    \centering
    \includegraphics[width=0.8\textwidth]{contents/part-2/fig2/challenger-rp2040.jpg}
    \caption{iLabs Challenger RP2040}
    \label{fig:challenger-rp2040}
\end{figure}

The Challenger board itself is well suited for this project for several reasons.
It uses the compact Adafruit feather form factor, giving a board dimension of
just 5cm by 2cm, making it easy to mount in a small enclosure. The onboard Hope
RF96 LoRa modem is built directly into the board and the U.FL antenna connector
allows for the swapping of antenna's to different varieties. This board's LoRa
module is also set to transmit at a frequency of 868mhz which is a standard UK
frequency for LoRa and gives a good balance between range and bandwidth.

Another useful aspect of the board is the abundance of GPIO pins (20 in total)
allowing for a large number of sensors to be fitted to the board.

\subsubsection{Antennae}

The selection of antennae is one of the largest determinants of range and
reliability in the context of wireless communication systems \cite{khan2016}.
Initially I used a simple PCB antenna as shown in Figure~\ref{fig:pcb-antenna},
however as explained in the next chapter, the range of this was insufficient for
my use.

\begin{figure}[H]
    \centering
    \includegraphics[width=0.8\textwidth]{contents/part-2/fig2/basic-antenna.jpg}
    \caption{Low range PCB antenna}
    \label{fig:pcb-antenna}
\end{figure}

To improve overall range I switched to a more capable omnidirectional whip
antenna that was made specifically for the Challenger RP2040. The antenna is
tuned to perform best at the 868mhz frequency range - which is the range I was
using.

\begin{figure}[H]
    \centering
    \includegraphics[width=0.8\textwidth]{contents/part-2/fig2/good-antenna.jpg}
    \caption{iLabs whip style LoRa antenna (868mhz)}
    \label{fig:good-antenna}
\end{figure}

\subsubsection{Powering the node}

To allow for continuous operation away from power sources, I attached a 6W
Monocrystalline Silicon Solar Panel (Figure~\ref{fig:solar-module})to each of
the nodes. I also installed a solar power management module
(Figure~\ref{fig:solar-module}) onto the Challengers. This module moderates the
output from the solar panels which is at too high a voltage to directly power
the nodes. It also incorporates a battery that charges from the solar panel
output and provides power to the nodes when the solar power is not available.

\begin{figure}[H]
    \centering
    \includegraphics[width=0.5\textwidth]{contents/part-2/fig2/solar-panel-manager.jpg}
    \caption{Waveshare solar power management module (left), solar panel (right)}
    \label{fig:solar-module}
\end{figure}

\subsubsection{Sensor selection} \label{sec:sensor-selection}

\paragraph{Temperature and humidity sensor}

I used a DHT11 temperature/humidity sensor (Figure~\ref{fig:dht11}) for each
sensor node to provide basic readings. It was chosen for it's low cost,
availability and compatibility with both the RP2040 Challenger and CircuitPython
(via libraries). The sensor can be connected to the microcontroller using a
single GPIO pin as well as the usual power and ground pin.

\begin{figure}[H]
    \centering
    \includegraphics[width=0.7\textwidth]{contents/part-2/fig2/dht11.jpg}
    \caption{Waveshare DHT11 temperature/humidity sensor}
    \label{fig:dht11}
\end{figure}

\paragraph{Soil moisture sensor}

The main choice between soil moisture sensors is whether to use a capacitative
or resistive style sensor. In a resistance sensor the diodes must be bare plated
metal in order for resistance between each diode to be measured. However the
disadvantage of this is that bare metal corrodes in the presence of water, and
corrosion affects the accuracy of the sensor.  The benefit of capacitative
sensors is that the diodes are covered by a protective layer making them much
less susceptible to corrosion. For this reason I chose a capacitive sensor
(Figure~\ref{fig:soil-sensor}).


\begin{figure}[H]
    \centering
    \includegraphics[width=0.7\textwidth]{contents/part-2/fig2/soil-sensor.jpg}
    \caption{The Pi Hut capacitive soil moisture sensor}
    \label{fig:soil-sensor}
\end{figure}

\paragraph{Wind speed sensor (Anemometer)}\label{sec:anemometer}

The anemometer chosen (Figure~\ref{fig:wind-sensor}) was the DFROBOT wind speed
sensor. It balances value with construction quality as unlike many other cup
style sensors this is made of metal and rated for outdoor use. When I chose this
component I was unaware that the RS485 serial communication protocol that the
sensor used was not compatible with the Challengers UART protocol. Fortunately,
I was able to purchase an RS485 to UART serial converter that allowed the
devices to communicate.

\begin{figure}[H]
    \centering
    \includegraphics[width=0.5\textwidth]{contents/part-2/fig2/wind-sensor.jpg}
    \caption{DFROBOT wind speed sensor}
    \label{fig:wind-sensor}
\end{figure}

A second issue was that the anemometer needed a 7V-24V input voltage to take
readings while the maximum voltage the Challenger can provide was 5V. To remedy
this I bought a 9V step up converter to boost the Challenger's voltage to the
required level.

\subsection{Gateway}

The components of the gateway consisted of a Challenger RP2040 with a whip
antenna. The board was connected to a Raspberry Pi 5 with 8GB of memory and an
onboard Wi-Fi radio. (Figure~\ref{fig:raspberry-pi}). This Raspberry Pi was
significantly more powerful than was needed but it was already available from
the university for use for this project. The Challenger receives and decodes
messages from the repeater. The Raspberry Pi reads the decoded LoRa message from
the serial output and then sends a POST response to the backend API which
inserts the data into a SQL database.

\begin{figure}[H]
    \centering
    \includegraphics[width=0.4\textwidth]{contents/part-2/fig2/raspberry-pi.jpg}
    \caption{Raspberry Pi 5 (8GB)}
    \label{fig:raspberry-pi}
\end{figure}



\section{Development and testing}

\subsection{Range tests}\label{sec:range-tests}

One of the most important tasks to carry out prior to deployment in the field
was testing the range of the devices.

For this experiment I took four Challenger RP2040s to The Downs, a large public
park in Bristol. Here I tested four different antenna configurations to compare
how well the signal travelled across an increasing distance. Signal strength can
be measured using the received signal strength index (RSSI), a measure of the
difference in signal from the transmitter to the receiver and measured in
decibels.

For the test, two Challengers were programmed as transmitters, sending an
example data packet similar to the data that would be generated by the sensor
nodes. One of the transmitters used a simple PCB antenna
(Figure~\ref{fig:pcb-antenna}) while the other used a higher range whip style
antenna (Figure~\ref{fig:good-antenna}). Then I programmed the remaining two
Challengers as receivers with one using a low range antenna and the other a long
range one.

This meant that four different antenna configurations could be tested
concurrently, as the receivers could pick up the signal from each transmitter.
Before performing the test I estimated that the maximum range of each
configuration would look like the below:

\begin{enumerate}
    \item Whip antenna to whip antenna (Likely best result)
    \item PCB antenna to whip antenna
    \item Whip antenna to PCB antenna
    \item PCB antenna to PCB antenna (Likely poorest result)
\end{enumerate}

The reason I predicted the PCB transmitter to whip receiver would outperform the
whip to PCB configuration is that improving receiver sensitivity (i.e. the
ability of the receiver to 'hear') is more efficient than increasing the
transmitters effective radiated power (how loud it shouts) \cite{simpulse25}.

Nine different distances from transmitter to receiver were tested, in 200m
increments starting from 0m as a baseline to a distance of 1600m
(Figure~\ref{fig:range-test-markers}). An important aspect in getting a
successful LoRa connection is whether there is line of sight between the
transmitter and receiver. Figure~\ref{fig:range-test-elevation} shows the
elevation profile of the test area, with the initial large dip being an
inaccuracy from google earth's topology data as the 0m point is near a cliff
edge. The lowest elevation point is at the starting point at around 83m above
sea level, while the highest point was at the 1km mark at around 94m making an
elevation range of 11m. My hypothesis was that signal would likely drop off or
stop entirely beyond this point as points beyond 1000m would be below the hill
line. This would effectively mean that the receivers would be in a signal shadow
point where the transmission waves would not be able to reach them. The only
possibility for signal to reach this area would be from reflections either from
nearby buildings or topology. The effects of reflections are virtually
impossible to account for in the real world and therefore no accurate
predictions could be made.

\begin{figure}[H]
    \centering
    \includegraphics[width=0.7\textwidth]{contents/part-2/fig2/range-test-markers.jpg}
    \caption{Google earth image of data collection points}
    \label{fig:range-test-markers}
\end{figure}


\begin{figure}[H]
    \centering
    \includegraphics[width=1\textwidth]{contents/part-2/fig2/range-test-elevation-profile.jpg}
    \caption{Elevation profile of test area (ignore large dip at start)}
    \label{fig:range-test-elevation}
\end{figure}

Unfortunately, at the time of the test a festival was being run between the
1200m and 1400m mark. The festival had a number of large tents and temporary
buildings constructed which very likely negatively affected the signal. Final
data for the range test is shown below.

\begin{figure}[H]
    \centering
    \includegraphics[width=0.9\textwidth]{contents/part-2/fig2/distance-graph.jpg}
    \caption{Graph to show signal loss from different receiver and transmitter configurations (higher is better)}
    \label{fig:range-test-graph}
\end{figure}

Unsurprisingly, the results show that the whip to whip antennas had the best
performance. It was the only configuration that consistently received signal all
the way to 1200m. In contrast with my prediction, the whip to PCB configuration
was the second best and even received a signal at 1200m; although this was less
consistent and had a lower signal strength compared to the whip-whip. I am
unsure why this was superior to the PCB-whip but clearly the increased effective
radiative power had a greater bearing on the RSSI than the sensitivity at the
receiver end. The PCB-whip managed only half the range at 600m and the PCB to
PCB managed just 200m, demonstrating its inadequacy for application in my
project.

\subsection{Battery tests}\label{sec:battery-tests}

There is no mains electricity available in the apple field so the nodes must be
able to operate without an external power source. Even with the use of a solar
panel the node must be able to work through the night and during periods of
dense cloud coverage where the solar panel will not have sufficient power to
keep the node running. This is where the use of a battery is essential as the
solar panel can charge the battery with excess energy during periods of excess
solar radiation, such as during midday hours, and then store this energy for
periods of low or no solar radiation.

However, while daylight will be of little concern during the summer months when
this dissertation is being written, there will be far fewer days of usable
sunlight over the winter period. As the UK has very high latitude, there is
large seasonal variation in the length of a day. For the town of Crediton (the
nearest town to \farmName), in the summer there are 16.5 hours of daylight while
in winter it receives only 7.6 hours; making one night 16.4 hours long.
Therefore the node must have a battery sufficiently large to power it for a
minimum of 16.4 hours with some additional charge to account for high cloud
cover during the early evening and morning period.

To see if the battery solution was sufficient for this environment a test was
run on a fully charged 18650 battery that was then connected to the solar power
manager to allow the voltage to be regulated up to the Challenger's 5V input
voltage.

A digital multimeter was installed between the solar power manager and the
Challenger's usb c input to view the voltage and current in real time as well as
running a timer for the test and a calculation of the number of watt-hours
consumed by the node. The node was then run with a full suite of sensors
attached. During the test the node used 80ma at 5V for a wattage of 0.4W.

The node ran until the battery would no longer discharge, with the final battery
life of the node being 19 hours and 17 minutes. The meter showed that the node
consumed 7.96Wh of power. Considering the battery capacity is 9 Wh it may seem
that the battery ran out too quickly. However, the solar power manager
documentation tells us that the battery boost efficiency - being the conversion
of battery voltage from native 3.7V to 5V output - is only 86\%. With that in
mind the predicted effective capacity is 7.74Wh, which is very similar to the
actual result.

The achieved result of 19 hours should be sufficient for running the node
continuously for much of the year as this compares to the longest night period
of 16.4 hours. However, during periods of heavy cloud cover in winter there is a
chance the node would not be able to charge the battery sufficiently in the day
to allow the node to run over night.

\subsection{Solar panel testing}\label{sec:solar-tests}

The solar panel is rated for 6W maximum power. This however would only
realistically be achieved under optimal conditions with a clear sky and full sun
around midday. This power rating was verified using a multimeter on such a day
where a voltage of 7.02V and an amperage of 0.87A was measured, giving a final
power of 6.1W.

As explained in the battery section above, the node consumed about 7.96Wh over a
19 hour and 17 minute period. We can therefore determine that the node requires
roughly 10Wh of energy per day to able to run continuously. It would be tempting
to then say that the node would need less than 2 hours of ideal sunlight to
operate for an entire day (being 6W * 2h = 12Wh). However we must also account
for the energy loss when converting from raw solar output to the regulated 5V
input that Challenger accepts.

The solar panel manager rates its conversion efficiency for solar at 78\%
meaning for each watt of solar energy received 22\% of this will be wasted as
heat when converted to the correct voltage. If we adjust for this loss we would
need effectively 12.8Wh of energy just to power the node for a day, which
equates to 2 hours and 8 minutes of ideal conditions.

An additional 9Wh is needed to charge the battery which if we adjust again for
solar efficiency factor we get 11.5Wh. This is an additional 1h:55m of ideal
sun, leaving a total sunlight requirement each day of 4h:03m.

\subsection{Hardware assembly and weatherproofing}

Since most of the hardware in this project contains exposed electronics, the
final assembled devices needed to be made wind and water resistant to prevent
failure. However the sensors and solar panel also needed to be exposed to
perform their function effectively - e.g. a solar panel must have a clear view
of the sun throughout the day. The final design therefore needed to balance
multiple conflicting purposes:

\begin{itemize}
    \item The core electronics (Challenger, solar manager, battery etc) must be
          kept dry, cool and away from wind which may damage electronics.
    \item The wind sensor must be exposed to the elements and securely fastened
          to withstand intense wind, it must also be kept high off the ground
          for more accurate readings.
    \item The solar panel must be exposed to the elements and be south facing at
          at an angle of 30-40 degrees to optimise solar efficiency.
    \item The soil moisture sensor must be inserted into the ground and as it
          has exposed electronics must be made waterproof.
    \item The antenna of the Challenger must be placed as high as possible (at
          least 1.5m from the ground).
    \item The temperature/humidity sensor must be exposed to the elements to
          ensure accurate readings but cannot be exposed to rain due to the
          exposed electronics on it.
\end{itemize}

Due to the conflicting nature of some of these requirements, the final device
would need to allow for positioning of different components at different
heights.

The final design therefore was to place most of the sensitive electronics inside
a waterproof box with cut-outs for wiring for external components. This was then
mounted onto a 2m pole to allow for a good antenna signal and more accurate wind
speed readings.

\subsubsection{Sensitive electronics}

An IP65 waterproof junction box (Figure \ref{fig:box-internals}) was selected to
house the non-sensor portion of the electronics. An IP rating measures the
Ingress Protection of a devices housing against water and dust defined by the
International Electrotechnical Commission. The first number of an IP code is a
measure of dust resistance while the second number is a measure of water
resistance. An IP code of 65 therefore means the box used here has perfect dust
resistance and can withstand water jets being sprayed at it from multiple
directions \cite{wiki-ip}. This level of protection should be more than adequate
to withstand the heavy rainfall it will experience.

\begin{figure}[H]
    \centering
    \includegraphics[width=0.5\textwidth]{contents/part-2/fig2/box-internals.jpeg}
    \caption{Open junction box showing internal wiring}
    \label{fig:box-internals}
\end{figure}

The Challenger microcontroller was inserted onto a half size breadboard and then
stuck to one side of the box. The antenna cable was routed to the top right
corner where a small hole was drilled to allow for the antenna to be mounted
externally. All the other components were then wired into the breadboard and
stuck down with double sided sticky pads to prevent damage during transport or
heavy winds. The bottom of the box had holes drilled for the anemometer, soil
moisture sensor, solar panel and temperature/humidity sensor respectively. To
prevent water ingress into these holes, sealant was deposited over the holes.

The entire box was attached to a plank of wood measuring roughly 25cm by 60cm.
This board holds all the sensors with the exception of the soil moisture sensor
which must be able to hang freely.

\subsubsection{Soil moisture sensor}

The sensor that required the most thought as to waterproofing and mounting was
the capacitative moisture sensor. The first challenge was the need for the soil
moisture sensor to be able to reach ground level while the microcontroller
collecting its readings was secured 2m above it. The solution to this was to
swap the small included wires with a 2.5m weatherproof cable containing three
cores (one for power, one for ground and one for data). Each cable core was then
soldered onto the sensor as in figure ~\ref{fig:solder-soil}, allowing for the
sensor to be inserted into the ground.

\begin{figure}[H]
    \centering
    \includegraphics[width=0.25\textwidth]{contents/part-2/fig2/solder-soil.jpeg}
    \caption{Soldered soil moisture sensor}
    \label{fig:solder-soil}
\end{figure}

Waterproofing the sensor was necessary as while the bottom three quarters of the
board are designed to be inserted into soil, the part above this is made up of
exposed electronics that must not come into contact with water (Refer to figure
~\ref{fig:soil-sensor}). This means it was essential to waterproof this part of
the board.

To achieve this I followed an online tutorial on a hobbyist website
\cite{waterproof-sensor}. and painted the electronics using nail varnish. While
this sounds unconventional, nail varnish is a non-conductive compound that can
be easily painted over electronics using the brush included and so is quite a
popular low-cost method for waterproofing electronics. After this is applied the
portion of the board with the electronics is covered in heatshrink to form a
tight seal over the board, with a layer of nail polish on the edges to reduces
the chance of water ingress under the heat shrink's plastic.

\subsubsection{Other external components}

The DHT11 sensor was mounted inside a smaller IP55-rated junction box, which
provides the same resistance against rain as the larger box. To allow for more
accurate readings, small holes were drilled on the bottom panel of this so the
air temperature inside the box would better match the external
temperature/humidity. To further improve the sensor's accuracy, the box was
painted white to reflect sunlight, and a solar shield made from a cut and
reshaped aluminium can was mounted to the south facing side. This shield (see
red box in Figure \ref{fig:solder-soil}) helps to prevent the sensor being
exposed to direct sunlight, reducing temperature distortion while still allowing
airflow from the sides.

The solar panel was mounted below the main enclosure using the included
adjustable gimbal mount. It was fixed at an angle of approximately 40° from
horizontal, which balances solar efficiency throughout the year
\cite{cathcart_best-solar-panel_2025}. This angle is steeper than the optimal
summer setting but allows for improved performance during winter months when the
sun is lower in the sky. The gimble allows for rotation in all planes which is
useful in case the box itself cannot be mounted in a perfect southerly
direction.

The anemometer was secured to a separate length of wood, positioned
approximately 0.5 metres away from the main unit. This separation reduces
turbulence caused by the box and pole, leading to more accurate wind speed
readings. The sensor was screwed directly into the wood and positioned at a
height of 2m.


\begin{figure}[H]
    \centering
    \includegraphics[width=0.6\textwidth]{contents/part-2/fig2/annotated-node.jpg}
    \caption{Final node design}
    \label{fig:assembled-node}
\end{figure}

Figure \ref{fig:assembled-node} shows the final assembled node. From top to
bottom, the purple box shows the antenna, the blue box shows the anemometer on
its separate arm, the green box contains the electronics, the red box is the
solar shield that surrounds the temperature/humidity sensor and the orange box
shows the solar panel. Not pictured is the 2m wooden post that extends the
height of the node.

\subsubsection{Gateway}

\begin{figure}[H]
    \centering
    \includegraphics[width=0.4\textwidth]{contents/part-2/fig2/gateway.jpeg}
    \caption{Completed gateway}
    \label{fig:gateway-final}
\end{figure}

The gateway node (Figure \ref{fig:gateway-final}) required less careful
engineering as there was no requirement for it to be waterproofed or mounted
externally. As this would eventually be placed inside a private home, the aim
was to make it minimally imposing so a design similar to an internet router was
chosen.

The two devices inside the box were the Raspberry Pi and the Challenger. A
plastic box was used with cut outs for the Pi's outputs and power cable, with an
additional cut out on the lid for the Challenger's antenna. The Pi and
Challenger were then secured to the box's base with velcro tape, as both devices
would potentially need to be removed for trouble shooting.

\section{Deployment}

\subsection{Test deployment at home}

Before setting up the nodes at the farm it was important to run the equipment in
an area where repairs and updates could be more easily performed. The nodes were
therefore placed at different locations in my garden and ran continuously for a
period of (ADD PERIOD).

\subsubsection{Challenges and solutions from test deployment}

One of the first challenges from the test deployment was the fact the batteries
on the nodes were not quite sufficient to allow 24/7 readings. On the fourth day
of the run, after an overcast evening the night before, one of the nodes stopped
transmitting at around 6am - just before sunrise. To remedy this another battery
was fitted to each of the end nodes (but not the repeater as this had much lower
battery consumption from the lack of any sensors). This doubling of battery
capacity would make loss of power during the night much less likely. The fact
the solar panels could charge the battery fully by mid morning pointed to the
fact that solar capacity was sufficient but battery capacity was not.

A related problem was that when the node died it did not start repeating again
after this despite solar power returning and the battery getting recharged. I
discovered that when the microcontroller detected brownout - being a sudden dip
in voltage - the device would enter a safeboot mode. In this safeboot mode the
default code.py file would not automatically be run and instead the device would
create and use a safemode.py file that without modification would essentially
run infinitely unless the reset button on the device was pressed or the device
was powered off and on again.

Clearly this behaviour was not appropriate for field deployment where brownouts
would potentially occur whenever the battery was discharged completely. Even
with the additional battery capacity provided by a second battery there would
still be a high chance of this occurring on particularly dark days during
winter, where continuous uptime could not be guaranteed.

To remedy this, I modified the safemode.py file using a template I found in the
circuitpython documentation. This version of the safemode.py file was made
specifically for the case I was using i.e. remote solar powered projects where
manual reset of the board could not be achieved easily. Instead of entering an
infinite loop this safemode would (DESCRIPTION OF SAFEMODE).

A seperate issue was the behaviour of the temperature-humidity sensor when
exposed to direct sunlight. While the junction box used to encase the sensor was
painted white to reduce any heat transfer from solar radiation, the readings
taken on very sunny days showed a large delta of around 5 celsius between the
sensor readings and local air temperature readings that could not be explained
with the existence of a microclimate.

To reduce this effect an additional solar shield was made to surround the
sensor. Consisting a wooden frame and aluminium shield to block light falling on
the junction box. While this method did help to reduce the delta at midday to
around 3 celsius there was still clearly quite a pronounced effect. So to
further reduce any warming the solar panel was mounted to the rear of the node.
This would allow the temperature sensor to face north instead of south which in
the northern hemisphere would result in much reduced solar warming to the
sensors box as the sun tracks over the southern area of the sky in the UK.

\subsection{Deployment in the field}

\subsection{Placement of nodes}

\part{Software development}

\section{Programming the hardware}

The challenger boards can be programmed in a few different languages. The most
common of these are C (including C++) and python. I decided to program the
boards in python due to the large availability of sensor libraries written in
this language. Programming in C, while potentially more efficient, would likely
have meant I would need to write my own libraries and functions to get many of
the sensors to function properly and therefore was not selected.

The RP2040 processor of the challenger uses relatively low power, so the
versions of python for the chip tend to be stripped down to accommodate this.
The two main python versions available are circuit python and micropython. They
both have extensive libraries and work with all the sensors I have selected.
With no major difference in their functionality, I settled on circuitpython as
this was the version recommended by iLabs (the makers of the challenger board)
and included a library that supports challenger as well as example code.

\subsubsection{The nodes}

\subsubsection{The repeater}

\subsubsection{The gateway}

\subsection{LoRa settings configuration}

The most fundamental part of the program for each of the nodes in the LoRa
network was that they could communicate with each other. This required the
careful matching of LoRa configuration settings between them.

  {ADD LoRa settings}

LoRa has many parameters that must be aligned for two devices to communicate
successfully. These include:

\begin{enumerate}
  \item \textbf{Spreading factor:}
  \item \textbf{Frequency:} Must match across devices; this project uses
        868\,MHz, the UK LoRa ISM band.
  \item \textbf{Bandwidth:} Determines how wide the signal is, I am using
        125\,kHz.
  \item \textbf{Coding rate:}
  \item \textbf{Transmit power:} Affects the power and therefore range. This
        must be set within legal limits (e.g., 14\,dBm in the UK).
\end{enumerate}

\subsubsection{Compliance with regulatory limits on radio power}\label{sec:lora-limit}

The UK has strict regulations on the usage of radio transmitters under the Ofcom
ISM band rules. For the 868mhz band, the maximum effective radiated power that
can be used is 25mW. This corresponds to a transmit power of roughly 14dB.

Additionally, the UK has rules on duty cycle rates. This is essentially how long
radio signals are permitted to be on air. For example at a spreading factor of
12 a message may take approximately 1 second to send. The duty cycle limit in
the UK is 1\%, meaning you may only transmit for 1\% of the time on a given day.
1\% of a day is 864 seconds. This means to stay in line with UK regulations only
864 messages could be sent on a given day - or roughly 1 message every 2
minutes.


\subsection{Sensor nodes}

\subsection{Repeater}

\subsection{Gateway}

\subsection{Remote diagnostics and error handling}

Since all the sensors would not be accessible easily errors needed to be
reported remotely to diagnose faults and problems. Unfortunately, while the
sensor nodes and repeater are technically network connected with LoRa there is
no easy way to send firmware updates over the air without developing a program
to accept and create new files over LoRa (FACT CHECK THAT), which was outside
the scope of a three month dissertation. Therefore instead the nodes would need
to have robust error handling written into their programs.

Diagnostics on the gateway node was luckily much simpler as this was powered on
24/7 and network connected. (EXPLANATION OF TAILWIND AND RVC VIEWER)


\section{Overview of software design}

\section{Backend: Database and SQL API}

For the database, I chose Render as a good low cost option for hosting the web
service as it offers a free trial period, constant uptime and backups. The free
version of web services by Render (render.com), provides a limited set of
resources at zero cost: 256 MB of RAM; 0.1 share of a CPU and 1 GB of storage.
There are also limits on outbound bandwidth, build pipeline minutes and instance
hours. A single instance of a POSTGRES database is available on this service -
which is free for the first 30 days and then costs \$5 a month. These resources
should be sufficient for this project as only small amounts of data are
involved.

The web service is running an API I have written, which will receive data from
the raspberry pi used as the gateway. The data is sent using a POST request
with a JSON body which is parsed, processed and added to the database by the
API. Data can be retrieved for display by the frontend using a GET request.

The database design is very simple with a single table holding all the data so
the choice of database management system is not crucial. Having said that,
POSTGRES is a well established and reliable relational database management
system that complies with ANSI SQL standards and most importantly for this
project is supported by Render.

The 1 GB storage limitation allows for (x) rows of data with the current
database design.

This solution should scale well. Additional tables can be added to the POSTGRES
data base for additional sites. And in the unlikely event that the volumes
exceed the Render limits it is easy to pay more for additional resources.

\section{Frontend: Agriscanner webapp}

Overview of front-end

\section{Forecasting with machine learning}\label{sec:machine-learning}

Once the webapp and database were operational I moved on to developing machine
learning models that would attempt to forecast the microclimate datapoints for
the following 48 hours based on the general weather forecast data for the
region.
\subsection{LightGBM}

LightGBM is a popular machine learning algorithm developed by Microsoft. I chose
LightGBM as it offers a good balance between performance and training efficiency
compared to similar models such as XGBoost \cite{saha2025}.

Both LightGBM and XGBoost are known as 'Gradient Boosted Machines'. This is a
technique in machine learning that effectively combines many weaker machine
learning models (called weak learners) to create a single highly accurate model.
These models are excellent for tabular data (such as my node and forecast data)
and regularly beat out competing learning algorithms \cite{tuychiev2023}.
LightGBM is also compatible with the m2cgen library that allows the final model
to be converted to JavaScript format, allowing it to work within my webapp. All
these factors made LightGBM a good algorithm for my use case.

To train the machine learning models I installed Python with the LightGBM and
m2cgen libraries as described above on my personal laptop. I then also installed
pandas for data handling purposes.

\subsection{Machine learning process}

I aimed to create ten separate machine learning models, one for each node (node
1 and node 2) and sensor reading (temperature, humidity, wind speed, gust speed
and soil moisture).

Training machine learning models requires inputs (referred to as “features”) and
outputs (referred to as “targets”).  For my models, the features came from the
OpenWeather past \textbf{actual} weather data and the targets came from the node
sensor readings. While training, the ML model can "see" both features and
targets. Figure \ref{fig:machine_diagram} below illustrates steps in the
training process.

\begin{figure}[H]
    \centering
    \includegraphics[width=1\textwidth]{contents/part-3/fig3/machine-learning-diagram2.jpg}
    \caption{Infographic showing steps for training with LightGBM. This uses the temperature target for illustration, for other sensor readings the target was changed to that reading}
    \label{fig:machine_diagram}
\end{figure}

The following steps were followed to train each of the ten models.

\begin{enumerate}
    \item Prepare the dataset: A single cleaned dataset was created by matching
          timestamps between the api weather data and the sensor node data. As
          API readings are taken every 10 minutes and node readings every 1
          minute, this meant that 9/10 node readings were discarded. The final
          dataset was roughly 1,400 rows.
    \item Define the feature set and target data: The feature set from the
          weather API and targets from the node data were defined, and
          unnecessary columns discarded. The database timestamp field was
          transformed into sine and cosine representations of day and year. This
          is necessary when training on a time-series data set as the algorithm
          must be able to understand the cyclical nature of time. For example,
          using raw timestamps would incorrectly suggest to the algorithm that
          the times of 23:00 on day 1 and 00:00 on day 2 are 23 hours apart
          rather than just 1 hour.
    \item Split the dataset into training data (80\%) and validation data
          (20\%): The data is split by time so the training data consists of the
          first 80\% of the rows and the validation data the last 20\%.  This
          data is then supplied to the model.
    \item Run the iterative training model: For each iteration, the model looks
          at the inputs (training features) and the correct answers (training
          target) of the training rows, and determines where it is getting
          incorrect outputs. It builds a small decision tree that specifically
          aims to correct those mistakes on the training rows and adds that tree
          into itself so its predictions change a little. It then applies the
          updated model to the validation inputs (validation features) and
          compares those predictions to the validation answers (validation
          target) —to see how well the model would do on new "unseen" data. The
          validation data are never used to build the tree; they are only used
          to check the accuracy of the model. If the validation check shows no
          improvement after a number of iterations, the training stops and the
          model keeps the version that performed best on validation.  The
          process will perform a minimum of 50 iterations. I set the maximum
          number of iterations to 250 to prevent the models getting too large,
          as each iteration increases the model size substantially (The humidity
          model is over 40,000 lines long in JavaScript format for example).
\end{enumerate}

Once trained, the final models were uploaded to the backend and the inputs then
came from the OpenWeather \textbf{forecasted} weather data for the next 48 hours
as explained in the following section.


\subsection{Machine learning deployment}

Once the ten models had been trained, they were uploaded to the web server.  I
wrote an automated function on my backend that provides the models with
datapoints from the OpenWeather \textbf{forecast} data for the next 48 hours,
and updates this data every ten minutes to adjust the predictions as the
forecast changes. The outputs from the models are recorded as hourly predictions
for each datapoint in a JSON file, which is requested by the frontend software
and used to display a line graph of predicted values for the next 48 hours.

\begin{figure}[H]
    \centering
    \includegraphics[width=0.8\textwidth]{contents/part-3/fig3/model_diagram.png}
    \caption{Infographic showing how final model is used on the webapp}
    \label{fig:model_diagram}
\end{figure}


\part{Evaluation}

\section{Hardware evaluation}

\subsection{Qualitative discussion of weather station performance}

\subsubsection{Battery life, solar power and outages}

The nodes have had significantly more downtime than expected based on the energy
budgets in Sections~\ref{sec:battery-tests} and~\ref{sec:solar-tests}, despite
doubling the battery capacity during deployment. The primary cause is site
placement, with both nodes in a north-facing garden bounded by 2.5\,m fencing
and a two-storey house immediately to the south. As a result, they receive no
direct sunlight until 09:00, and Node 1 is shaded again by 15:00 due to the
shadow of a nearby garage. This leaves only \(\approx\)5\,h of direct sunlight,
which is far below the amount assumed in the design of the nodes.

Outages seem to only affect the sensor nodes; the repeater has had zero downtime
since installation. This is consistent with its much lower energy budget from a
lack of any peripheral sensors, which reduces average energy draw.

The limitations of the current location, plausibly explain the observed outages.
Because these shading conditions are unlikely in an open field, these outages
are not of primary concern for the intended deployment. However, it highlights
the system's sensitivity to late sunrise and early sunsets from terrain. For
example if the actual deploy site was shaded by a hill or some tall trees late
in the day, which I had not properly considered in the design phase.

\subsubsection{Weatherproofing}

The summer of 2025 is set to be one of the hottest and driest on record with
August specifically receiving roughly half its average rainfall
\cite{uor2025summer}. Consequently, from the 15th of August to the 28th of
August the weather stations saw only short spells of rain, making an analysis of
their weather proofing hard to review up to this point. 

Fortunately, from the 28th August to the 2nd September Chipping Sodbury - where
the nodes are currently placed - received roughly 3.4cm of rain according to
weather forecasts. Throughout this period there has been no evidence of water
ingress.

Beyond ingress testing, the nodes experienced elevated ambient temperatures
(exceeding \SI{32}{\degreeCelsius}) and showed no signs of overheating. These
results are encouraging, although longer-term deployment that includes colder
weather and sustained heavy rain is needed to comment conclusively on their
overall robustness.

\subsubsection{Range}\label{sec:range-eval}

As reported in section ~\ref{sec:range-tests} the challenger microcontrollers
and antennas used in this configuration are able to reach at least 1200m of
range in a real world situation. The conditions in that test were not ideal for
LoRa as both the receiver and transmitter were positioned low to the ground at
around 1-1.5m, additionally line of sight was broken at around 1000m which
severely diminishes the performance of any radio communication system. There
were also buildings between 1200m and 1400m that blocked LoRa signals entirely.

In the intended deployment of my full weather network, the use of an additional
repeater would significantly improve range. In a flat terrain setting then the
use of a repeater would clearly double the range, however if terrain is hilly a
repeater can increase range by even larger multiples. The below graphic
demonstrates the problem of hills when it comes to LoRa. As a LoRa signal
propagates from the transmitter and collides with a hill the majority of the
signals energy is reflected off the hill meaning it cannot reach the gateway.

\begin{figure}[H]
    \centering
    \includegraphics[width=0.9\textwidth]{contents/part-4/fig4/no-repeater.png}
    \caption{Illustration of LoRa radio propagation between node and receiver with blocking terrain}
    \label{fig:no-repeater}
\end{figure}

If a repeater is placed on the top of the hill not only would the LoRa signal
reach the gateway receive signal but the repeaters increased elevation would
give a significant boost to its range as the total area with a line of sight to
the repeater would be vastly increased.

The range that this repeater allows is what sets the design of my network apart
from commercial options and is discussed in more detail in the next section..

\begin{figure}[H]
    \centering
    \includegraphics[width=0.9\textwidth]{contents/part-4/fig4/repeater.png}
    \caption{Illustration of LoRa radio propagation with a repeater included}
    \label{fig:repeater}
\end{figure}

\subsubsection{Reset behaviour}\label{sec:reset-behaviour}

As covered briefly in Chapter~\ref{sec:deployment}, the reset behaviour of the
challenger is inconsistent after a brownout or complete power outage.
Essentially if the power of the sensor nodes is cut off at night then it is not
guaranteed that the node will restart the code.py program in the morning. This
is in spite of the fact that I modified the safemode.py behaviour of
circuitpython with the recommended code for remote solar power applications. As
this ultimately appears to be a firmware issue there is not an obvious or easy
solution for this. 

With enough time I would probably rewrite all of the challenger software in
Micropython and see if that firmware behaves in a more useful way on power
outage. Unfortunately the issue was picked up fairly late in development and it
was unrealistic to perform a refactor of this scale with the time left.

\subsection{Quantitative comparison with commercial alternatives}

I will now compare the performance of my LoRa weather station to prebuilt
options available to purchase. A breakdown of costs for Agriscanner is
included in Appendix~\ref{app:cost-nodes}.

\begin{table}[H]
\centering
\small
\renewcommand{\arraystretch}{1.2}
\begin{tabularx}{\textwidth}{l >{\raggedright\arraybackslash}X
  >{\raggedright\arraybackslash}X >{\raggedright\arraybackslash}X
  >{\raggedright\arraybackslash}X >{\raggedright\arraybackslash}X}
\hline
 & \textbf{Agriscanner Network} & \textbf{SenseCAP
 S2120\cite{pihut:sensecap-s2120-2025}} & \textbf{Decentlab Eleven
 Parameter\cite{alliot:decentlab-eleven-2025}} & \textbf{HOBO weather station
 kit\cite{weathershop:hobo-rx3000-2025}} & \textbf{SparkFun Arduino weather
 kit\cite{pihut:sparkfun-2025}} \\

\hline
Number of sensors & 4 & 8 & 11\textsuperscript{*} & 6 & 7 \\
Sensor accuracy & Hobbyist & Hobbyist & Professional & Professional & Hobbyist
\\
Communication type & LoRa & LoRa & LoRa & Mobile network & WiFi \\
Update frequency & 1 minute & 1 hour & 10 minutes & 1 hour & 1 minute \\
Readings per hour & 60 & 1 & 6 & 10 & 60 \\
Power source included & Yes & Yes & Yes & Yes & No \\
Power source & Solar & Solar & Solar & Solar & -- \\
Batteries recharge? & Yes & No & No & Yes & -- \\
Reported battery life & replace $\sim$ 3 years & 154 days & Several months &
replace 3--5 years & -- \\
Reported range & -- & 2--10\,km & 2--10\,km & Anywhere with 4G & -- \\
Estimated range & 2.4--20\,km & 1.2--10\,km & 1.2--10\,km & -- & 10--50\,m \\
IP rating & $\sim$ IP65 & IPX6 & IP66 & IP66 & None \\
Ongoing payment? & -- & -- & -- & Yes -- mobile plan & -- \\
Ongoing costs p.a & \pounds{}0 & \pounds{}0 & \pounds{}0 & \pounds{}132 &
\pounds{}0 \\
Cost per sensor node & \pounds{}177 & \pounds{}287 & \pounds{}3{,}272 &
\pounds{}4{,}138 & \pounds{}130 \\
Cost per repeater & \pounds{}93 & \pounds{}0 & \pounds{}0 & \pounds{}0 &
\pounds{}0 \\
Cost per gateway\textsuperscript{**} & \pounds{}66 & \pounds{}122 & \pounds{}122
& \pounds{}0 & \pounds{}0 \\
Battery cost p.a.\textsuperscript{***} & \pounds{}7 & \pounds{}10 & \pounds{}30
& \pounds{}20 & \pounds{}0 \\
\textbf{Total cost\textsuperscript{****}}  & \textbf{\pounds{}520} &
\textbf{\pounds{}706} & \textbf{\pounds{}6{,}696} & \textbf{\pounds{}8{,}418} &
\textbf{\pounds{}260} \\
\hline
\end{tabularx}

\vspace{0.25em}
\textit{\footnotesize * Sensors missing from Agriscanner: Solar radiation,
rainfall, barometric pressure, vapor pressure, dew point, wind direction, tilt
sensor, lightning strike count / distance} \\
\textit{\footnotesize ** For SenseCAP and Decentlab models the lowest cost
gateway available is sensecap m2 at £122} \\
\textit{\footnotesize *** Battery cost assumptions are detailed in
Appendix~\ref{app:battery-assumptions}.} \\
\textit{\footnotesize **** Includes sensors, repeater, gateway, and estimated first-year battery + ongoing costs.}
\caption{Comparison of weather-station options.}
\label{tab:commercial-comparison}
\end{table}

\subsubsection{Benefits of my weather station}

\begin{itemize}
  \item \textbf{Cost:} At £520 the Agriscanner weather stations are the lowest
  cost option among waterproof, long-range systems. While the SparkFun kit is
  cheaper, it is not suitable for outdoor deployment due to its exposed
  electronics and, since it relies on WiFi, its range would be insufficient in
  an agricultural setting. It will therefore not be discussed further. The two
  professional systems (DecentLab and HOBO) are over ten times the price of my
  system and are therefore not comparable.
  
  The only system with similar characteristics under £1,000 that I could find is
  the SenseCAP device. However, with a cost per node roughly 50\% higher, my
  nodes still provide better value. If additional nodes were deployed, this
  price difference would become even more significant.

  \item \textbf{Frequency of readings:} My sensor nodes collect and transmit
  readings every minute, providing the Agriscanner webapp with effectively live
  data. By contrast, most alternative options only provide hourly readings,
  leaving actual conditions between measurements unknown.

  \item \textbf{Battery recharging:} My nodes use rechargeable 18650 batteries,
  which I conservatively estimate will last around three years (the same battery
  chemistry as mobile phones) before requiring replacement. Of the alternatives,
  only the HOBO system features a rechargeable battery. Both SenseCAP and
  DecentLab rely on disposable alkaline batteries, requiring regular replacement
  and therefore more frequent maintenance. 

  \item \textbf{Range:} As explained in Section~\ref{sec:range-eval}, my weather
  station network benefits from the inclusion of a repeater. This enables
  significantly greater range than the other models could provide, particularly
  in hilly terrain.
  
  I also doubt whether either the SenseCAP or DecentLab systems could achieve a
  substantial range improvement under the same test conditions. In the UK, LoRa
  transmission power is subject to a strict regulatory cap (see
  Section~\ref{sec:lora-limit}). Since my system already operates at this
  maximum power, commercial products cannot exceed it. Even with a superior
  antenna, transmit power would need to be reduced proportionally, as antenna
  gain also counts towards the effective radiated power. It is therefore highly
  unlikely that these alternatives would achieve a range advantage sufficient to
  offset the benefits of a repeater in my system.
\end{itemize}

\subsubsection{Drawbacks of my weather station}

\begin{itemize}
  \item \textbf{Number of sensors:} One of the clearest drawbacks of the
  Agriscanner system is the lower sensor count: only four sensors compared with
  seven to eleven on competing systems. However, the Challenger microcontrollers
  in my sensor nodes still have spare GPIO ports, so adding additional sensors
  would be straightforward. The cost of many sensors is also not prohibitive
  (for example, a UV index sensor is only \pounds{}6).
  \item \textbf{Sensor accuracy:} Professional systems use higher-grade sensors
  with tighter accuracy specs. For example, the DecentLab unit specifies air
  temperature accuracy of $\pm0.6\,\si{\celsius}$, whereas the DHT11 used in
  mine is only rated to about $\pm2.0\,\si{\celsius}$.
  \item \textbf{Reset behaviour:} As noted in Section~\ref{sec:reset-behaviour},
  the node firmware can behave unreliably after a full battery drain. While I
  cannot purchase the other devices in this list to confirm this directly, it is
  unlikely that the other units would exhibit the same behaviour, especially
  because their firmware will built specifically for remote solar powered
  applications.
  \item \textbf{Other factors:} The other devices bring other non-tangible
  benefits besides their technical specification. For example,  warranties,
  formal testing, vendor support etc. Professional systems commonly offer
  long-term maintenance and firmware updates beyond the initial purchase, which
  my prototype solution obviously does not offer. 
\end{itemize}

\subsection{Conclusion on hardware performance}

While the weather station network has shown promising results as a low cost and
long range solution, the unreliable reset behaviour holds it back from being
100\% deployment ready. So far the weatherproofing of the external devices has
been encouraging, though testing has been limited due to unusually dry weather
and a longer deployment in wetter and colder conditions is needed. The outages
seen during the current deployment appear to be an issue specific to the private
garden they are in and would be much less likely in an open field environment;
however, this does emphasise the need for more extensive site surveys prior to
installation. Overall, if the reset behaviour can be addressed and a small
number of additional sensors are added, the complete system would be well suited
for long term unattended deployment on a farm.

\section{Web-app evaluation}

This Chapter will evaluate the usability of the Agriscanner webapp using results
from a System Usability Scale (SUS) survey that was carried out via online form
in late August.

\subsection{Procedure and test subjects}

Before the main section of the survey, participants were asked to consent to
standard University of Bristol privacy wording (see
Appendix~\ref{app:sus-survey}). To avoid stricter GDPR handling I did not record
identifiable information (e.g. name, email, age), which may have also encouraged
more honest responses. I also made sure to ask what device they were viewing the
project on (either mobile or desktop) as I wanted to test whether there was any
difference in usability between the two.

Participants were then asked to perform a set of four representative tasks in
the webapp. Each task involved collecting a piece of information from a
specified chart, such as finding the temperature for a particular node at a a
specified time (tasks list shown in Appendix~\ref{list-of-tasks}). Once the
tasks were completed, I asked participants to submit these answers via
multiple-choice questions.

Next, the survey presented a standard 10-question SUS, which is a Likert-scale
questionnaire commonly used to report on the usability of software systems
\cite{brookeSUS1995}. The SUS gives a score from 0 to 100, where a higher score
indicates that a system is more usable. A "good" SUS score is generally regarded
as anything above 68 \cite{sauro2016quantifying}.

Finally, there was an optional textbox for participants to fill out asking for
feedback on bugs or features they would like to see. I included this to generate
user stories for future development (Section~\ref{sec:user-story}).

By asking participants to perform the same tasks and collecting their answers, I
ensured everyone completed a valid interaction with the webapp before rating it.
This standardises the experiment context and reduces uncontrolled variance that
can arise from an open “try out my website” approach. Because I recorded the
task responses, I could also compare them to the correct answers, providing an
additional objective metric (task success) to complement the SUS.

In total, 15 participants completed the survey. Participants were recruited
mainly from friends, family and other students on my course. Clearly, this small
and non-random sample limits the generalisability of the results, so any
interpretation of the following sections should be treated more as an indication
of the webapp's usability rather than conclusive evidence.

\subsection{Hypotheses}

\subsubsection{Hypothesis 1}
The sample SUS score will be greater than the benchmark value of 68

\textbf{Statistic to test:} One-sample, right-tailed Wilcoxon signed-rank
test\footnote[1]{For rationale and sources on statistical testing refer to
Figure \ref{fig:appendix-note-1} in the Appendix}.

\textbf{Null and alternative hypotheses:}
\[
H_0:\ m = 68 \qquad\text{vs}\qquad H_a:\ m > 68,
\]
where \(m\) is the population median.

\subsubsection{Hypothesis 2}
The SUS scores for users on mobile will not differ significantly from those on
desktop.

\textbf{Statistic to test:} Two-sample, two-tailed Mann-Whitney U
Test\footnotemark[1].

\textbf{Null and alternative hypotheses:}
\[
H_0:\ \text{the two populations are equal} \qquad
H_a:\ \text{the two populations are not equal.}
\]

\subsection{Results}

This section summarises the key results from the survey. Refer to
Table~\ref{tab:raw-sus} in the Appendix for data per participant.

\begin{figure}[H]
    \centering
    \includegraphics[width=0.3\textwidth]{contents/part-4/fig4/box-whisker.png}
    \caption{Box and whisker plot showing range of SUS scores from participants}
    \label{fig:box-whisker}
\end{figure}

\begin{table}[H]
  \centering
  \small
  \begin{minipage}[t]{0.48\textwidth}
    \centering
    \begin{tabular}{l r}
      \hline
      Metric & Value \\
      \hline
      Maximum     & 100.0 \\
      Quartile 3  & 95.0 \\
      Median      & 90.0 \\
      Mean        & 87.7 \\
      Quartile 1  & 86.3 \\
      Minimum     & 67.5 \\
      \hline
    \end{tabular}
    \captionof{table}{Summary statistics for SUS}
    \label{tab:sus-metrics}
  \end{minipage}\hfill
  \begin{minipage}[t]{0.48\textwidth}
    \centering
    \begin{tabular}{l c r}
      \hline
      Group & N & Mean SUS \\
      \hline
      Desktop & 7 & 84.3 \\
      Mobile  & 8 & 90.6 \\
      \hline
    \end{tabular}
    \captionof{table}{Mean SUS score by device}
    \label{tab:sus-by-device}
  \end{minipage}
\end{table} 

The chart in Figure~\ref{fig:box-whisker} shows the tight distribution of values
between the first and third quartiles, with most participants giving favourable
ratings. There were also a smaller number of lower ratings shown as outliers,
with a minimum score 67.5.

Table~\ref{tab:sus-metrics} shows the key metrics from the survey. The mean
score of 87.7 is well above the benchmark of 68 and indicates a high perception
of usability for the Agriscanner webapp. A one-sample Wilcoxon signed-rank test
confirmed that the median SUS score was significantly greater than 68
(\(p=0.0004\)), allowing for the rejection of the first null hypothesis \(H_0\).

Table~\ref{tab:sus-by-device} indicates that participants using mobile devices
gave slightly higher usability ratings (90.6) than those on desktop (84.3). To
confirm whether this difference was significant, a two-sample Mann-Whitney U
test was conducted. The test found no significant difference between the two
groups with (U=13.5,cv = 10, p=0.105). Therefore, the second null hypothesis
\(H_0\) could not be rejected at the 5\% significance level. Based on this there
is no significant difference between the two scores.


\begin{table}[H]
  \centering
  \begin{tabular}{r c r}
    \hline
    Question no. & Incorrect answers & Percentage incorrect\\
    \hline
    1 & 3 & 20\%\\
    2 & 1 & 7\% \\
    3 & 1 & 7\% \\
    4 & 0 & 0\% \\
    \hline
    \multicolumn{2}{r}{Overall percentage correct} & 92\% \\
    \hline
  \end{tabular}
  \caption{Number of incorrect answers for task quiz}
  \label{tab:correct-metrics}
\end{table}

Performance in the assigned tasks was generally good
(Table~\ref{tab:correct-metrics}), with participants on average answering
correctly 92\% of the time. Questions 2, 3 and 4 were answered correctly by
almost all participants, with only one participant failing on 2 and 3. The first
question proved the most difficult to perform accurately with 3 candidates
failing. I decided to plot the SUS score given against the percentage of correct
answers (see~\ref{app:correlation-sus}) however with an $R^2$ value of 0.0066
there was effectively zero correlation between the two; even if there was some
correlation the small sample size and ceiling effect from so many respondents
getting perfect scores would make this difficult to detect.

\subsection{Discussion and limitations of results}

With a very high mean (87.7) and median (90) SUS score, the results here suggest
a very high degree of usability. Additionally, the high task completion rate
adds further quantitative evidence that users could use the website's functions
without explicit guidance.

With that said there a number of important limitations to these findings. As
already mentioned, the sample is small and non-random so the results are not
necessarily representative of the wider population. Because I knew many
participants personally, social-desirability bias is likely to have inflated
ratings. Likewise, a high proportion of computer-science students in the sample
means these participants may have been more familiar with web interfaces than
the general public, which could also have contributed to the higher SUS scores.

Not all the data were positive however: Two participants gave scores close to
the benchmark level used to indicate good usability and one respondent gave a
score below this point. With such small sample it is important not to dismiss
these as outliers. 

Fortunately, the three lower scoring participants left useful comments
(Figure~\ref{fig:low-sus-feedback}) that point to the usability issues they
encountered. Additional comments from other participants are shown in
Figure~\ref{fig:high-sus-feedback} and recurring themes are summarised in
Table~\ref{tab:feature-requests}.

The most frequently mentioned issue was frustration with the (calendar and
date-range selection, and finer time selection on the chart), discoverability of
the compare feature, and clearer presentation of soil-moisture readings. Other
suggestions include data export, quicker switching between measurement types,
and minor layout/responsiveness fixes for chart axes. These categories provide
useful guidance for the next development cycle.

\subsection{User stories from SUS survey} \label{sec:user-story}

A number of user stories came out of the survey that can be used as future
requirements:

\begin{table}[H]
  \centering
  \small
  \begin{tabularx}{\textwidth}{>{\RaggedRight\arraybackslash}p{0.28\textwidth}
  >{\centering\arraybackslash}p{1.5cm} >{\RaggedRight\arraybackslash}X}
    \hline
    Category & Mentions & User story \\
    \hline
    Calendar with date-range selection & 4 & As a user I want the ability to
    select date ranges so that I can quickly jump to past dates without
    repeatedly clicking through days. \\
    \hline
    Clearer way to find compare graph & 2 & As a user I want the compare graph
    to be more explicitly signposted so that I can easily find it when trying to
    compare nodes \\
    \hline
    Human readable soil moisture readings\textsuperscript{*} & 1 & As a user I
    want an option to view soil moisture in a human readable format (e.g.
    wet/dry) so that I can understand what the sensor values mean. \\
    \hline
    Tooltip granularity on chart & 1 & As a user I want smoother control when
    using the tooltip without snapping to a particular minute so that I can
    select more precise times reliably. \\
    \hline
    Way to export data  & 1 & As a user I want a way to an export weather data
    to a CSV file so that I can have offline access to it. \\
    \hline
    Switch between measurements without going back & 1 & As a user I want a
    control from within each sensor section (temperature, humidity etc) that
    allows me to switch between measurement types quickly so that I don't have
    return back to the home page between each navigation. \\
    \hline
    About page & 1 & As a user I want an "About" page describing the project and
    the data sources so that I understand the context of the data. \\
    \hline
    Specific issue with chart & 1 & As a user I want the Y axis to align with my
    screen properly so that no text is cut off and I can read the labels \\
    \hline
  \end{tabularx}
\noindent\textsuperscript{*}\small This has now been implemented on the webapp
\caption{Compiled feature requests from survey}
  \label{tab:feature-requests}
\end{table}

\subsection{General discussion on webapp}



\section{Machine learning model evaluation}

This chapter evaluates the predictive performance of my machine learning model.
The model's forecasts are benchmarked against both actual sensor data and the
predictions from a simpler alternative model, before discussing the results.

\subsection{Models being compared}

The three models that I compared against the actual sensor readings are included
below.

\begin{enumerate}
    \item{General forecast weather from OpenWeather} 
    \item{Proposed machine learning model (Chapter \ref{sec:machine-learning})} 
    \item{Alternative model - an adjusted version of the OpenWeather using a mean average adjustment}
\end{enumerate}

\subsection{Procedure}

To evaluate the models I compared the results from all three models over the
same 48 hour period (which is the limit of the OpenWeather forecast). The
measurements started at 00:00 on Saturday, 30 August 2025, and concluded at
23:00 on Sunday, 31 August 2025. 

The OpenWeather forecast and machine learning prediction were collected
automatically by my backend (using node\_cron and endpoints, see Section
\ref{sec:building-api}) and stored in a temporary database table I made
specifically for this evaluation.

The alternative model was made by comparing the average temperature, humidity
and wind speed differences between my raw sensor data and OpenWeather current
weather readings from my database. An average difference was then calculated
between the two datasets and applied to the raw OpenWeather forecast. The aim of
this was to adjust the forecast readings to be closer to the sensor readings. To
make the comparison fair with the machine learning model, I used data from
15--27 August (the same as the ML's training data) so that the alternative model
would not have a larger amount of data. The purpose of including the alternative
model was to assess whether a simple bias correction model was any different to
my more complex machine learning procedure.

\subsection{Charts and results}

As the weather data for the selected period was very similar between both nodes
I have decided to only show charts for node 1 as it is representative of node 2
as well. I have included a table containing the Mean Absolute Error (MAE) and
Root Mean Squared Error (RMSE) for the data in these charts in Appendix
\ref{app:ml-stats}.

\begin{figure}[H]
    \centering
    \includegraphics[width=0.95\textwidth]{contents/part-4/fig4/temperature-graph.png}
    \caption{Chart comparing temperature forecast models}
    \label{fig:temperature-chart}
\end{figure}

While Figure \ref{fig:temperature-chart} does show that the machine learning
model (orange) roughly tracked the observed readings, it does not appear to be
substantially more accurate than either the general forecast (green) or the
alternative model (blue). The relative MAE (see Appendix \ref{app:ml-stats}) of
the machine learning model here is 2.9\% while the general forecast is only
slightly higher at 3.4\%\footnote{MAE scores should be interpreted as follows: a
higher score means that the error compared to the actual sensor reading was
greater. A lower score indicates a closer (better) relationship}, meaning the ML
model is only marginally better than its input data at predicting temperature.
The alternative model performs better with an MAE of 1.3\%. Both the ML and
alternative models predicted a higher peak temperature on both days than either
the sensor data or the forecast, suggesting that the data from 15th - 27th
August may not have been a representative sample and is biased towards higher
temperatures.

\begin{figure}[H]
    \centering
    \includegraphics[width=1\textwidth]{contents/part-4/fig4/humidity-graph.png}
    \caption{Chart comparing humidity forecast models}
    \label{fig:humidity-chart}
\end{figure}

In Figure \ref{fig:humidity-chart} the machine learning model performed
particularly poorly on average with an MAE OF 14.9\% (General forecast: 5.5\%).
The alternative model also performs worse with MAE of 11.1\%. While these
metrics paint a poor picture of the model, I would point to the section of the
chart around 12:00 31/08. Here we can see that the machine learning model is the
only one of the models to successfully predict a spike in humidity at midday.
While the magnitude of this spike is not correct it does offer some tentative
evidence that the model can more accurately predict some data patterns that a
the general forecasts cannot predict.

\begin{figure}[H]
    \centering
    \includegraphics[width=1\textwidth]{contents/part-4/fig4/wind-speed-graph.png}
    \caption{Chart comparing wind speed forecast models}
    \label{fig:wind-chart}
\end{figure}

With wind speed (Figure \ref{fig:wind-chart}) the ML model performs better
relative to either the general forecast or the adjusted forecast used in the
alternative model. This is reflected in an MAE of 43.6\% for the machine
learning model versus 381.9\% and 264\% for the general forecast and alternative
model respectively. With that said, the machine learning model does not show any
"hump" at midday meaning that the model is not predicting the pattern of wind
speed well. This suggests the training data had fewer windy days than the 48
hours evaluated here

\begin{figure}[H]
    \centering
    \includegraphics[width=1\textwidth]{contents/part-4/fig4/soil-graph.png}
    \caption{Chart comparing soil moisture predicted by machine learning model versus actual readings}
    \label{fig:soil-chart}
\end{figure}

The soil moisture chart (Figure \ref{fig:wind-chart}) only has the data for the
machine learning model and the observed data because the general forecast and
alternative models do not capture this metric. The large MAE of 162\% here is
not surprising at all however - the model's training data was all from an
extremely dry period of time and therefore the model does not adjust soil
moisture for recent precipitation. The days before the 48 hour period used in
these charts had included a lot of rain so the soil was still still wet from
this. However the model still expected dry soil as no connection between
rainfall and damp soil had been established in the training rounds.

\subsection{Discussion of results}

While the results for the machine learning model are mixed, there are some
bright spots. For example, the model still broadly tracks the actual sensor
readings for both temperature and humidity. With respect to temperature, while
the MAE of 2.9\% was not the best score, in Celsius this represents an error of
less than 0.5$^\circ$C. If we only look at the temperature error for node~2, the
error was even smaller at around 0.2$^\circ$C (node~1 was 0.7$^\circ$C), and it
was the best performer of the three in this context.

Additionally, the humidity chart shows that the model can detect some patterns
that the general forecast does not, as seen in the brief reversal in humidity on
day~2. Wind speed and soil moisture were clear wins for the model, but mostly
due to the highly inflated figures for wind in the forecast and the lack of any
soil moisture forecast data.

Despite this, on average the alternative model marginally outperformed the
machine learning model in temperature, and both models were surprisingly worse
than the general forecast in the humidity context. This raises the question of
whether machine learning is "worth it" if a computationally simpler model
performed at around the same level.

I would argue that yes, using machine learning was worth it, for two reasons:
The first is around the training data: the machine learning model here was not
trained on nearly enough data to provide a fair evaluation of its
microforecasting abilities compared to the other models. This is most obvious
from the soil moisture readings, where the model had no prior rain data meaning
it would have been impossible to predict that soil moisture levels would be
lower after rainfall. In general weather conditions from the 15th to the 27th
are not particularly representative of the conditions after this point, as the
weather has been cooler and cloudier past this point with higher average
humidity.

The second point is that the machine learning approach is capable and likely to
improve, unlike the other solutions which are fundamentally not predicting the
microclimate. For example the alternative model is essentially a blunt
correction to the general forecast, as the dataset grows this approach would
become increasingly poor. If in summer the microclimate tends to be hotter and
in winter it tends to be cooler then using a simple mean average correction
would not only result in no change to the forecast but also a poor prediction in
both seasons.

Hence I believe that with a longer deployment and more data it is highly likely
that the machine learning approach would show much greater accuracy than the
other approaches, and is one of the interesting avenues for future work (Section
\ref{sec:future-work}).

\section{Future work}

Future improvements for weather station and app

\section{Closing remarks}

Brief conclusion


% References
\newpage
\addcontentsline{toc}{section}{References}
\printbibliography

\appendix
\newpage
\part{Appendices}

% Override spacing just for appendix
\makeatletter
\patchcmd{\section}{\clearpage\vspace*{3cm}\thispagestyle{plain}}{\clearpage\thispagestyle{plain}}{}{}
\titleformat{\part}[block]
{\normalfont\centering\bfseries} {\LARGE \partname\ \thepart}{1em}{\Huge}
[\thispagestyle{plain}\clearpage]
\makeatother

%TC:ignore
\section{Breakdown of costs}\label{app:cost-nodes}

\begin{figure}[H]
    \centering
    \includegraphics[width=1\textwidth]{contents/appendix/fig5/table.jpg}
    \caption{Table of component costs}
    \label{fig:cost-table}
\end{figure}

\begin{figure}[H]
    \centering
    \includegraphics[width=0.8\textwidth]{contents/appendix/fig5/chart.png}
    \caption{Chart to visualise relative cost of components}
    \label{fig:cost-chart}
\end{figure}

\section{Interview excerpt with Small Brook Farms
owners}\label{sec:small-brook-interview}

Interviewer: So you would need a weather station to observe very local weather,
I expect. What you're saying is that the BBC website weather is not necessarily
relevant to you?

Speaker 1: No, No , so there's another site which is in Sanford and we can have
conversations - we're what? - 5 miles apart 10 miles? 

Speaker 2: Yeah. So the other side of that hill there like a mile away you get a
different sort of weather, but even on this side of the [apple orchard] as
opposed to that side of the [apple orchard] like the wind can be less than
whatever else, its very localised. But you know you if you then go on that side
of the valley, it doesn't rain on this side. So like to be actually useful,
yeah, [weather monitoring] sort of has to be [based on] the farm.

Source: Transcript no.28 of site visit (9 May 2025)

\section{Battery cost assumptions}\label{app:battery-assumptions}
\begin{itemize}
  \item \textbf{Agriscanner Network:} Replace 5 (2 per sensor node, 1 for
  repeater) Li-ion batteries every 3 years at a cost of \pounds{}20
  \(\Rightarrow\) \(\approx\)\,\pounds{}7 per annum.
  \item \textbf{SenseCAP S2120:} Three AA batteries per node, replaced twice a
  year (total 12 batteries) \(\approx\)\,\pounds{}10 per annum.
  \item \textbf{Decentlab Eleven Parameter:} Two C batteries per node, replaced
  four times per year (total 16 batteries) \(\approx\)\,\pounds{}30 per annum.
  \item \textbf{HOBO weather station kit:} Replace lead-acid for each node
  battery every 4 years at a cost of \pounds{}80 \(\Rightarrow\)
  \(\approx\)\,\pounds{}20 per annum.
\end{itemize}

\section{System usability survey}\label{app:sus-survey}

\begin{figure}[H]
    \centering
    \includegraphics[width=0.9\textwidth]{contents/appendix/fig5/sus_survey.png}
    \caption{SUS survey wording}
    \label{fig:survey-wording}
\end{figure}

List of tasks for users:

\begin{enumerate}\label{list-of-tasks}
  \item To the nearest degree, what was Node 2's temperature at 14:00 18 August
  2025?
  \item To the nearest 1\,m/s, what was Node 1's gust speed at 13:00 26 August
  2025?
  \item Go to wind speed and change the graph to compare mode. What colour is
  Node 2 gusts represented by?
  \item What is the forecast for humidity at 06:00 tomorrow for Node 1?
\end{enumerate}

\section{Additional SUS materials}\label{app:correlation-sus}

\begin{figure}[H]
    \centering
    \includegraphics[width=0.9\textwidth]{contents/appendix/fig5/answer-sus-correlation.png}
    \caption{Graph to show SUS score vs percent of correct answers ($R^2 = 0.0066$)}
    \label{fig:sus-correlation}
\end{figure}

\begin{table}[ht]
  \centering
  \begin{tabular}{r r r r}
    \hline
    responder\_num & sus\_score & correct\_answers & type\\
    \hline
    1  &  90.0  & 75\%  & desktop\\
    2  &  92.5  & 100\% & mobile\\
    3  &  95.0  & 100\% & mobile\\
    4  &  70.0  & 100\% & desktop\\
    5  &  90.0  & 100\% & mobile\\
    6  &  90.0  & 100\% & desktop\\
    7  &  87.5  & 50\%  & desktop\\
    8  &  95.0  & 100\% & mobile\\
    9  &  95.0  & 75\%  & mobile\\
    10 &  95.0  & 100\% & desktop\\
    11 &  72.5  & 100\% & desktop\\
    12 &  67.5  & 75\%  & mobile\\
    13 & 100.0  & 100\% & mobile\\
    14 &  90.0  & 100\% & mobile\\
    15 &  85.0  & 100\% & desktop\\
    \hline
  \end{tabular}
  \caption{Raw SUS scores and correct answer percentage from survey}
  \label{tab:raw-sus}
\end{table}

% Requires: \usepackage{float} Also requires: \usepackage{amsmath}
\begin{figure}[H]
  \centering
  \begin{minipage}{0.9\textwidth}
    \begin{quote}
    ``I found it very hard to get to a specific time on the graph - the
    granularity of the cursor moving seemed to make it hard to choose my time -
    I ended up with 14:01 for the first question as I couldn't get the cursor to
    stay at 14:00. But as a weather nerd I loved it.''
    \end{quote}
 \vspace{8pt}
    \begin{quote}
    ``I found it annoying having to click to get to the required date. Also I
    don't understand from the website alone what the project is about or where
    the data is from - maybe an about page would be nice : )''
    \end{quote}
 \vspace{8pt}
    \begin{quote}
    ``Sorry wasn’t sure where the compare graph is but the website looked really
    nice!''
    \end{quote}
  \end{minipage}
  \caption{Feedback from participants with lower SUS scores}
  \label{fig:low-sus-feedback}
\end{figure}

\begin{figure}[H]
  \centering
  \begin{minipage}{0.9\textwidth}
    \begin{quote}
    "The soil moisture readings surprised me - I expected them to go up with
    rain (increased moisture) but they went down.  I think this is
    counter-intuitive and there should either be some explanation of the what
    the measurements mean, or preferably there should be an option to convert to
    a more human understandable description like very dry/dry/slightly
    moist.../wet/very wet/saturated"
    \end{quote}
 \vspace{8pt}
    \begin{quote}
    "When selecting the date (specifically when clicking the arrows to move
    forwards and backwards in time) it would be useful to have a calendar
    display to travel to past dates quicker."
    \end{quote}
 \vspace{8pt}
    \begin{quote}
    "Extra features: - Date picker instead of having to navigate past each day -
    Export feature to export data in bulk (e.g. to a csv) - Ability to select a
    time period (specific dates) instead of just a day view"
    \end{quote}
 \vspace{8pt}
    \begin{quote}
    "I think when switching dates, it should support selecting a date from the
    calendar rather than only moving forward or backward to the nearest dates."
    \end{quote}
 \vspace{8pt}
    \begin{quote}
    "Being able to navigate directly between measurements (e.g. temp, humidity
    etc.) while on [sic]"
    \end{quote}
 \vspace{8pt}
    \begin{quote}
    "Finding the "compare" option was the hardest part. But didn't take long." 
    \end{quote}
 \vspace{8pt}
    \begin{quote}
    "Great website- simple layout and easy to use!"
    \end{quote}
 \vspace{8pt}
    \begin{quote}
    "No bugs seen, but on wind graph some of the y axis text was slightly cut
    off on my screen. I'm on a laptop"
    \end{quote}
  \end{minipage}
  \caption{Feedback from participants with higher SUS scores}
  \label{fig:high-sus-feedback}
\end{figure}

\begin{figure}[H]
  \makebox[\textwidth][r]{ \fbox{
      \begin{minipage}[c][8cm][c]{0.85\textwidth}
        \raggedright

        Both the Mann Whitney U and Wilcoxon Signed Rank test were selected for
        this data because the SUS score is derived from Likert items. Likert
        items are ordinal (e.g. strongly disagree) and so nonparametric
        statistical testing is typically preferred
        \cite{bobbitt_mann-whitney_2022}\vspace{8pt}

    The Mann Whitney U statistical test score was calculated in an Excel sheet I
    made using the procedure described in \cite{bobbitt_mann-whitney_2022} and
    the result cross-verified using the website in \cite{socscistats} which is
    also the source of the p-value. Results: U = 13.5, critical value = 10
    p=0.10524. Parameters: two-tailed, 0.05 significance\vspace{8pt}
    
    Due to the complexity of calculating it, the one-sample Wilcoxon Signed Rank
    Test score was calculated in Excel using a modified template from the source
    in \cite{peterstatistics_wilcoxon_2025} and cross-verified with the website
    in \cite{statsBlue}. Results: \(W^{+}=119\) z = 3.3361, critical value =
    1.6449 Parameters: right-tailed, 0.05 significance \end{minipage} } }
  \caption{Note on statistical testing}
  \label{fig:appendix-note-1}
\end{figure}

\section{Nielsen's nine usability heuristics
\cite{nielsen1990heuristic}}\label{app:usability-heuristics}

\begin{itemize}
  \item Simple and natural dialogue
  \item Speak the user's language
  \item Minimize user memory load
  \item Be consistent
  \item Provide feedback
  \item Provide clearly marked exits
  \item Provide shortcuts
  \item Good error messages
  \item Prevent errors
\end{itemize}


\section{How LoRa works}\label{app:lora-explained}

As this paper is not a technical study of radio communication I will opt for a
brief summary of the principles behind LoRa. With this in mind I have based much
of the information from the excellent video lecture in \cite{visualelectric2021}
that itself draws upon the paper in \cite{vangelista2017}.

The reason LoRa modulation is different to traditional modulation techniques is
the use of a "chirp" as the key to transmitting packets. A more traditional
technique might involve a frequency shift key; that is a single frequency
represents several bits. These unique frequencies are called symbols as they
represent data, like letters in the alphabet. In the below graph we see three
simplified symbols that represent binary values, the combination of these
symbols makes a packet:

\begin{figure}[H]
  \centering
  % left image
  \begin{minipage}{0.48\textwidth}
    \centering
    \includegraphics[width=0.4\linewidth]{contents/part-1/fig1/frequencysymbols.png}
    \\[4pt]
    {\small (a) Frequency shift symbols} \end{minipage}\hfill
  % right image
  \begin{minipage}{0.48\textwidth}
    \centering
    \includegraphics[width=1\linewidth]{contents/part-1/fig1/packet.png}
    \\[4pt]
    {\small (b) Packet}
  \end{minipage}
  \caption{ Traditional radio modulation with frequency symbols}
  \label{fig:freq-and-packet}
\end{figure}

In traditional modulation, symbols always have flat unchanging frequency (as can
be seen from the fact the wave separation never changes). LoRa symbols instead
have changing frequencies that have a waveform like the below.

\begin{figure}[H]
    \centering
    \includegraphics[width=0.15\textwidth]{contents/part-1/fig1/lorawavelength.png}
    \caption{LoRa symbol showing changing frequency ("Up-chirp")}
    \label{fig:lora-wave}
\end{figure}

This change in frequency is what gives the wave form the name "chirp". If
traditional frequency symbols were thought of a sound they would be similar to
morse code beeps while LoRa would be more similar to siren or \textit{chirp}ing
bird. LoRa has both rising and falling chirps (up-chirps and down-chirps).

Different LoRa symbols are then distinguished by the point in time of a
discontinuity. LoRa symbols are always delivered over a known length of time, so
different symbols include a reset back to the starting frequency at a different
point in time.

A way to graphically show this discontinuity in LoRa and compare it to frequency
shift modulation is by using instantaneous frequency graphs:

\begin{figure}[H]
  \centering
  % left image
  \begin{minipage}{0.48\textwidth}
    \centering
    \includegraphics[width=0.9\linewidth]{contents/part-1/fig1/traditional-wavechart.png}
    \\[4pt]
    {\small (a) Frequency shift symbols} \end{minipage}\hfill
  % right image
  \begin{minipage}{0.48\textwidth}
    \centering
    \includegraphics[width=0.9\linewidth]{contents/part-1/fig1/lora-wavechart.png}
    \\[4pt]
    {\small (b) LoRa symbols, note shifting discontinuity}
  \end{minipage}
  \caption{ Comparison of frequency shift and LoRa modulation }
  \label{fig:freq-vs-lora}
\end{figure}

Once these symbols hit the receiver, the receiver must work out which symbol it
was. In frequency shift modulation this is achieved by performing a correlation
test against every symbol received. However this is computationally difficult
and requires a low signal-to-noise ratio to work effectively. The benefit of
chirps is that due to a mathematical transformation that will not be further
discussed (Fast Fourier transform), the correlation can be computed much more
easily and with less powerful hardware.


\section{IoT enabling technologies} \label{app:enabling-tech}

This supplementary section lists the key technological developments that have
are contributed to the viability of my project.

\begin{enumerate}
  \item Efficiency improvements in microchips - breakthroughs in microchip
  fabrication have led to smaller more efficient chips with improved
  performance.
  \item Lithium-Ion batteries improvements - continuous improvements in the
  energy density of lithium-ion batteries has made it possible to power devices
  for long periods without mains power.
  \item Low-power long-range radio - new radio communication techniques such as
  LoRa allow for data transmission over several kilometres using a fraction of
  the power required by traditional mobile or Wi-Fi technologies.
  \item Affordability of solar panels - since 1970 the price of solar panels has
  decreased to 1/500th of its original cost \cite{economist2024} making solar a
  viable power source for IoT systems.
  \item Growth of hobbyist embedded systems - since the release of accessible
  platforms such as Arduino in 2005, the growth of hobby level embedded systems
  has lowered the barrier to entry to create IoT systems.
  \item Accessible cloud computing and hosting - Cheap and available web hosting
  has allowed application level systems to be more easily developed.
\end{enumerate}

\section{CircuitPython sensor node 1 code example}\label{app:sensor-code}

\begin{multicols}{2}
\begin{lstlisting}
import time
import board
import busio
import digitalio
import adafruit_rfm9x
import adafruit_dht
import analogio


DEVICE_ID = 1
RADIO_FREQ_MHZ = 868.0
WIND_REQUEST = bytes([0x02, 0x03, 0x00, 0x00, 0x00, 0x01, 0x84, 0x39])

# Initialise LORA radio and settings
try:
    spi = busio.SPI(board.RFM95W_SCK, MOSI=board.RFM95W_SDO, MISO=board.RFM95W_SDI)
    cs  = digitalio.DigitalInOut(board.RFM95W_CS)
    rst = digitalio.DigitalInOut(board.RFM95W_RST)
    rfm9x = adafruit_rfm9x.RFM9x(spi, cs, rst, RADIO_FREQ_MHZ)

    rfm9x.tx_power = 13
    rfm9x.spreading_factor = 7
    rfm9x.signal_bandwidth = 125_000
    rfm9x.coding_rate      = 5
    rfm9x.enable_crc       = True
    rfm9x.implicit         = False

except Exception as e:
    rfm9x = None
    print("ERR: Lora module", e)
    
# Initialise UART and settings    
try:
    uart = busio.UART(tx=board.GP16, rx=board.GP17, baudrate=9600, timeout=0.2)
except Exception as e:
    uart = None
    print("ERR: UART", e)
    
# Initialise the soil moisture sensor    
try:    
    moisture_pin = analogio.AnalogIn(board.A0)
except Exception as e:
    moisture_pin = None
    print("ERR: moisture pin", e)

# Initialise DHT11 temperature humidity sensor
try:
    dht_device = adafruit_dht.DHT11(board.A1)
except Exception as e:
    dht_device = None
    print("ERR: dht11", e)
    
# Returns raw soil moisture reading
def get_raw_moisture(pin):
    try:
        raw = pin.value   
        return raw
    except:
        return None

# Returns wind speed sensor using MODBUS protocol
def get_wind_speed():
    if uart:
        try:
            uart.write(WIND_REQUEST)
            time.sleep(0.2)
            response = uart.read(16)
            if response:
                if len(response) >= 5 and response[0] == 0x02 and response[1] == 0x03:
                    value_raw = response[3] << 8 | response[4]
                    return value_raw / 10.0
                else:
                    print("Bad format error")
                    return None
            else:
                print("No response error")
                return None
        except:
            print("Unknown error")
            return None
    print("Bad UART error")
    return None

def find_err(value):
    if value is None:
        return "ERR"
    else:
        return value

counter = 0

while True:
    
    t_sum = h_sum = w_sum = 0.0
    s_sum = 0
    t_n = h_n = s_n = w_n = 0
    min_counter = 0
    w_max = 0
    
    while min_counter < 10:
        if dht_device:
            try:
                t_reading = dht_device.temperature
                if t_reading is not None:
                    t_sum += float(t_reading)
                    t_n += 1
            except Exception:
                pass
            try:
                h_reading = dht_device.humidity
                if h_reading is not None:
                    h_sum += float(h_reading)
                    h_n += 1
            except Exception:
                pass
        s_reading = get_raw_moisture(moisture_pin)
        
        if s_reading is not None:
            s_sum += s_reading
            s_n += 1
            
        w_reading = get_wind_speed()
        
        if w_reading is not None:
            w_sum += float(w_reading)
            w_n += 1
            if w_reading > w_max:
                w_max = w_reading
        
        min_counter += 1
        time.sleep(6)
        
        
    if t_n != 0:    
        t_avg = t_sum/t_n
    else:
        t_avg = None
    if h_n != 0:    
        h_avg = h_sum/h_n
    else:
        h_avg = None
    if s_n != 0:    
        s_avg = s_sum/s_n
    else:
        s_avg = None
    if w_n != 0:
        w_avg = w_sum/w_n
    else:
        w_avg = None
        w_max = None
    
    payload = f"{DEVICE_ID},{find_err(t_avg)},{find_err(h_avg)},{find_err(s_avg)},{find_err(w_avg)}, {find_err(w_max)},{counter}"
    try:
        rfm9x.send(payload.encode("utf-8"))
        print(f"Sent packet {payload}")
    except Exception as e:
        print("Failed to send packet: ", e)

    counter += 1
\end{lstlisting}
\end{multicols}

\section{Typescript API endpoint code for node data
insert}\label{app:api-endpoint}
\begin{multicols}{2}
\begin{lstlisting}
app.post('/api/database/insert-node-data', async (req: Request, res: Response) => {
  const {
    device_id,
    packet_id,
    temperature,
    humidity,
    soil_moisture,
    wind_speed,
    gust_speed,
    rssi0,
    rssi1,
    snr0,
    snr1,
  } = req.body;

  try {
    const query = `INSERT INTO node_data 
      (node_deployment_id, farm_id, packet_id, temperature, humidity, 
      soil_moisture, wind_speed, gust_speed, rssi0, rssi1, snr0, snr1, node_name)
      VALUES (
      (SELECT id FROM node_deployment WHERE node_name = $11 ORDER BY ts DESC LIMIT 1),
      (SELECT farm_id FROM node_deployment WHERE node_name = $11 ORDER BY ts DESC LIMIT 1),
      $1,$2,$3,$4,$5,$6,$7,$8,$9,$10,$11);`;
    const values = [
      packet_id,
      temperature,
      humidity,
      soil_moisture,
      wind_speed,
      gust_speed,
      rssi0,
      rssi1,
      snr0,
      snr1,
      device_id
    ];
    await pool.query(query, values);
    res.status(200).json({ status: 'success' });
  } catch (error) {
    console.error(error);
    res.status(500).json({ error: 'failed to insert to database' });
  }
});
\end{lstlisting}
\end{multicols}

\section{Python code to train machine learning model}\label{app:ml-code}
\begin{multicols}{2}
\begin{lstlisting}
import pandas as pd
import numpy as np
import lightgbm as lgb
import joblib as jl
import m2cgen as m2c
from sklearn.metrics import mean_absolute_error, mean_squared_error

VALIDATION_PERCENT = 0.2
NODE_ID = 1
TARGET = 'temperature'

NODE_PREFIX = 'node_' + f'{NODE_ID}'
TARGET_COL = NODE_PREFIX + '_' + TARGET

# Read the csv and make it into pandas dataframe
data_frame = pd.DataFrame(pd.read_csv(NODE_PREFIX + '_training_data.csv'))

# Remove sensor data other than the target
cols_to_remove = ['ts']
for col in data_frame.columns:
    col_name = str(col)
    if col_name != TARGET_COL and col_name.startswith(NODE_PREFIX):
        cols_to_remove.append(col_name)

data_frame = data_frame.drop(columns = cols_to_remove)

# Set the split for training and validation data
cut_off_index = int(round((1- VALIDATION_PERCENT) * len(data_frame), 0))

# Define the data 
training_data = data_frame.iloc[:cut_off_index]
validation_data = data_frame.iloc[cut_off_index:]

# Define the features used for prediction
features = ['day_sin', 'day_cos', 'year_sin', 'year_cos', 'temp', 'pressure', 
'humidity', 'uvi', 'clouds', 'wind_speed', 'wind_gust', 'rain_1h', 'snow_1h']

# Initialise LGBM with following parameters
training_model = lgb.LGBMRegressor(
    n_estimators=250, 
    learning_rate=0.05
)

# Fit the model using features and target data 
training_model.fit(
    training_data[features],
    training_data[TARGET_COL],
    eval_metric='rmse',
    eval_set=[(validation_data[features], validation_data[TARGET_COL])],
    # Stop running training after 50 iterations of no improvement
    callbacks=[lgb.early_stopping(50), lgb.log_evaluation(20)]
)

# If it ends from best iteration get the date from this
best_iteration = getattr(training_model, "best_iteration_", None)
if best_iteration:
    prediction = training_model.predict(validation_data[features], num_iteration=best_iteration) 
else:
    training_model.predict(validation_data[features])

#Console log useful stats
print("Mean absolute error:", mean_absolute_error(validation_data[TARGET_COL], prediction))
print("Root mean squared error:", np.sqrt(mean_squared_error(validation_data[TARGET_COL], prediction)))
print("Features used:", training_model.booster_.feature_name())

#Export final model to JS
javascript_final_model = m2c.export_to_javascript(training_model)

with open(f'{TARGET}{NODE_ID}.js', 'w') as f:
    f.write(javascript_final_model)

    \end{lstlisting}
\end{multicols}

\section{Mean absolute error (MAE) and root mean squared error (RMSE) from
forecast comparison}\label{app:ml-stats}

\begin{figure}[H]
    \centering
    \includegraphics[width=0.9\textwidth]{contents/appendix/fig5/mae-rmse.png}
    \caption{General forecast is OpenWeather, Machine learning refers to my models, alternative refers to a mean adjusted general forecast model}
    \label{fig:mae-rmse}
\end{figure}

\begin{figure}[H]
    \centering
    \includegraphics[width=1\textwidth]{contents/appendix/fig5/rel-mae-rmse.png}
    \caption{Table with relative MAE and RMSE averaged across node 1 and 2}
    \label{fig:rel-mae-rmse}
\end{figure}

\section{Alternative model data}

The below figure was used to produce the alternative model by applying these
adjustments to the general OpenWeather forecast.

\begin{figure}[H]
    \centering
    \includegraphics[width=1\textwidth]{contents/appendix/fig5/rel-mae-rmse.png}
    \caption{Spreadsheet snippet showing average difference between general forecast and sensor readings between (15-27 August)}
    \label{fig:alt-data}
\end{figure}
%TC:endignore

\end{document}
