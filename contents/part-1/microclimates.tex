\section{Microclimates}

This section gives some academic background on microclimates with a focus on
their relevance to agriculture, and then looks at related attempts to predict
microclimate weather.

\subsection{What is a microclimate?}

A microclimate is generally understood as a set of distinct climatic conditions
within a small, localised area \cite{MetOffice2023}. The maximum size of a
microclimate is debated, but the World Meteorological Organisation (WMO) regards
it as occupying an area of anywhere from less than one metre across to several
hundred metres \cite{wmo2024}.  In practice, microclimates can occur in spaces
such as gardens, valleys, caves, or fields. Even human-made structures can
generate their own microclimates; for example, tall buildings can create
\emph{street valleys} that reduce wind flow and lead to the formation of
localised pockets of warmer air, which  can trap higher concentrations of
pollution from vehicle emissions \cite{yang2023}. Vegetation plays a critical
role in influencing microclimates. The addition of trees to an urban environment
can reduce air temperature by as much as \SI{2.8}{\degreeCelsius}
\cite{lai2019}.

\subsection{Microclimates in agriculture}

Microclimatic variations have profound implications for agriculture, as the
climate that crops are exposed to has an enormous impact on overall agricultural
yields. Indeed, farmers have modified the microclimate of crop fields for
millennia, a clear example of this being the use of fencing to reduce soil
erosion and damage to edible plants \cite{cleugh1998}. Therefore, the
relationship between microclimates and agriculture has been the subject of
extensive research, particularly as climate change introduces new threats to
food security.

A Danish study by Haider et al. investigated how agricultural pests and diseases
are influenced by microclimatic conditions. Temperature sensors were installed
in six different areas with distinct microclimates (such as hedges and cattle
fields). The data revealed that the daytime temperature in these microclimates
was significantly higher than those predicted by Danish weather forecasts. Using
these measurements, the researchers then estimated the incubation periods of
various pests and diseases, demonstrating that elevated temperatures in
microclimates could shorten incubation times and thus accelerate the risk of
disease outbreaks \cite{haider2017}.

Another important aspect of microclimates for farmers is how it can affect frost
risk - sudden and unpredictable drops below freezing in crop fields that damage
plants and are particularly common in spring. A principal factor for this is a
lack of "cold-air drainage". In areas with depressed topography, cold, dense air
can accumulate to form pockets where temperatures can be several degrees lower
than the surrounding landscape \cite{drepper2022}. These local conditions are
not captured by general weather forecasts highlighting the need for more precise
monitoring and development of effective early warning systems.

\subsection{Microclimate prediction using machine learning}

There have been a number of recent studies where machine learning has been used
to help predict micro climate conditions in agricultural settings. General
weather models operate at magnitudes between 1 and 10 km and microclimate
predictions require models that operate at scales of roughly 100m or less. Using
current numerical weather prediction models for micro-scale predictions is
computationally expensive \cite{blunn2024machine}, and these models have lower
accuracy rates than predictions using machine learning due to the inherent
complexity and non-linear nature of microclimates.

A number of the studies described here focus on comparing the accuracy achieved
by different types of machine learning, and highlighted a distinction between
machine learning (ML) and deep learning (DL).  They are both systems that learn
from data and make decisions based on datasets provided when the systems are
trained.  ML techniques include decision trees, regression and neural networks.
DL systems are a type of ML that uses neural networks with many layers and is
used to model more complex patterns with larger datasets. The table below
provides a summary of the studies included in this section


\begin{table}[ht]
      \centering
      \scriptsize
      \setlength{\tabcolsep}{6pt}
      \renewcommand{\arraystretch}{0.98}
      \caption{Selected summary of machine learning studies}
      \label{tab:forecast-studies}
      \resizebox{\textwidth}{!}{%
            \begin{tabular}{|>{\centering\arraybackslash}m{2.2cm}
                  |>{\centering\arraybackslash}m{3.2cm}
                  |>{\centering\arraybackslash}m{1.8cm}
                  |>{\centering\arraybackslash}m{3.2cm}
                  |>{\centering\arraybackslash}m{3.0cm}
                  |>{\centering\arraybackslash}m{3.0cm}|}
                  \hline
                  \textbf{Study}                                    &
                  \textbf{Location}                         & \textbf{Type of
                  learning}   & \textbf{Learning framework} & \textbf{Source of
                  forecast data}          & \textbf{Source of training target
                  data} \\
                  \hline
                  Agriscanner (This work)                           & Outside
                  rural - South Gloucestershire, UK & ML & LightGBM &
                  OpenWeather                               & Sensor readings \\
                  \hline
                  Kumar et al.\ (2021)\cite{kumar2021}              & Various &
                                                                    DL & LSTM &
                                                                    Not
                                                                    specified &
                                                                    Sensor
                                                                    readings \\
                  \hline
                  Zanchi et al.\ (2023)\cite{zanchi2023harnessing}  & Outside
                  rural; Italy, Lombardy & DL & Feed-forward neural network &
                  \makecell[c]{ERA5 \\ ARPA } & Sensor readings \\
                  \hline
                  Blunn et al.\ (2024)\cite{blunn2024machine}       & Outside
                  urban; London, UK                 & ML \& DL &
                  \makecell[c]{Multilayer Perceptron \\ Random Forest \\
                  XGBoost}                                          & UKV
                  weather data                          & Citizen weather
                  stations                                        \\
                  \hline
                  Abdelmadjid (2025)\cite{abdelmadjid2025enhancing} & Greenhouse
                  - Unknown                      & ML \& DL &
                  \makecell[c]{CNN-LSTM \\ LSTM (DL) \\ SVM-RBF \\ Prophet \\
                  LightGBM \\
                  XGBoost}                                          & Public
                  kaggle dataset                     & Public kaggle dataset \\
                  \hline
            \end{tabular}%
      }
\end{table}

A 2021 study by Kumar et al \cite{kumar2021} developed an ML framework called
DeepMC as a part of a Microsoft Research initiative. Their model is able to
predict a variety of climatic variables such as soil moisture, wind speed and
temperature using inputs from weather station forecasts and IoT sensors. They
were able to get up to 90\% accuracy with a 12-120 hour forecast range.

Zanchi et al \cite{zanchi2023harnessing} used physical modelling of local
terrain combined with DL to forecast the microclimate in the foothills of
Lombardy. The objective was to predict the local conditions at the meter-scale
as opposed to the 10km+ scale of regional and global weather forecasts. The
initial model combined data about the morphology of the local terrain and
weather forecast data to provide the input data for two feed-forward neural
networks. These neural networks were trained to predict the local weather
variables using data from 25 sensors deployed in the region being studied.  The
study demonstrated that local predictions were more accurate when using forecast
data from local weather stations (ARPA) as opposed to global climate datasets
(ERA5), but accuracy was still good either.

This study is of particular interest for this dissertation as it used IoT
sensors to generate the training targets for the neural networks and in this
regard is the most similar to my design. The paper also includes a discussion of
problems with reliability of the sensors - only 4 out of 25 units ran without
failure over the period of the study - and the need to clean the data to ensure
these failures were excluded from the training data. In my project, I came to
appreciate how time consuming the development of hardware is, and how many
unexpected issues can occur. As this project had a hard end date, this affected
the quantity of data I was able to collect and therefore ultimately impacted the
training of the machine learning models.

Blunn et al \cite{blunn2024machine} ran a study focussed on predicting
temperatures in urban environments during heatwaves, using data from eight
heatwaves in London, UK. They used data from the UKV - a high-resolution weather
forecasting model - and from citizen weather stations (CWS), made available via
the Weather Observations Website - a cloud platform where individuals can upload
data from their personal weather stations. The study tested a variety of ML
models including one DL model. The authors used a similar model training design
to that used in this dissertation. The models were trained on UKV variables
(i.e. a general forecast as with mine) and CWS variables (i.e. sensor
information) to bias correct the UKV readings and create a forecast prediction
model that could predict the CWS readings accurately (mean average error:
0.12\(^\circ\)C) compared with the general weather readings from UKV (mean
average error: 0.64\(^\circ\)C). The main point of difference to my own study
here is that I am using four different climate variables compared to just
temperature in the Blunn paper. Also CWS is publically available civilian
weather stations whereas I am using my own sensor data from a devices I have
built.

A very recent paper from Abdelmadjid et al 2025 \cite{abdelmadjid2025enhancing}
used online datasets from Kaggle to develop an ML tool to predict changes in
temperature and humidity within greenhouses in response to changes to external
weather conditions.  They used this data to test three ML models and three DL
models and selected the LightGBM ML model and the LSTM DL model as the best
performing models for prediction. The overall system design consisted of four
LSTM models feeding into the LightGBM model. This design resulted in 98.45\%
accuracy for temperature predictions and 99.61\% accuracy for humidity
predictions. This study informed my decision to use the LightGBM model for my
system as they found it  to be reliable and effective in capturing underlying
data patterns. 