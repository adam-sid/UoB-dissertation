\section{Microclimates}

A microclimate is generally understood as a set of distinct climatic conditions
within a small, localised area \cite{MetOffice2023}. The maximum size of a
microclimate is debated, but the World Meteorological Organisation (WMO) regards
it as occupying an area of anywhere from less than one metre across to several
hundred metres \cite{wmo2024}.  In practice, microclimates can occur in spaces
such as gardens, valleys, caves, or fields. Even human-made structures can
generate their own microclimates; for example, tall buildings can create
\emph{street valleys} that reduce wind flow and lead to the formation of
localised pockets of warmer air, which can also trap higher concentrations of
pollution from vehicle emissions \cite{yang2023}. Vegetation plays a critical
role in influencing microclimates. The addition of trees to an urban environment
can reduce air temperature by as much as \SI{2.8}{\degreeCelsius}
\cite{lai2019}.

\subsection{Microclimates in agriculture}

Microclimatic variations have profound implications for agriculture, as
localised conditions directly impact crop yields. The climate that crops are
exposed has an enormous impact on overall agricultural yields. Indeed, farmers
have modified the microclimate of crop fields for millennia, a clear example of
this being the use of fencing to reduce soil erosion and damage to edible plants
\cite{cleugh1998}. Therefore, the relationship between microclimates and
agriculture has been the subject of extensive research - particularly as climate
change introduces new threats to food security.

A Danish study by Haider et al. investigated how agricultural pests and diseases
are influenced by microclimatic conditions. Temperature sensors were installed
in six different pathogen habitats (such as hedges and cattle fields). The data
revealed that the daytime temperature in these microenvironments was
significantly higher than those predicted by Danish weather forecasts. Using
these measurements, the researchers then estimated the incubation periods of
various pests and diseases, demonstrating that elevated temperatures in
microclimates could shorten incubation times and thus accelerate the risk of
outbreaks \cite{haider2017}.

Another important aspect of microclimates for farmers is how it can affect frost
risk, that is sudden and unpredictable drops below freezing in crop fields that
damages plants, particularly common in spring. A principal factor for this is a
lack of "cold-air drainage". In areas with depressed topography, cold, dense air
can accumulate to form pockets of cold air where temperatures can be several
degrees lower than the surrounding landscape\cite{drepper2022}. These local
conditions may not be captured by weather forecasts highlighting the need for
more precise monitoring and development of effective early warning systems.

However, there is limited research on microclimates in the context of apple
orchards which is most relevant to this dissertation. A study conducted by
Bristol university took sensor measurements throughout an orchard's growing
season, revealing that conditions such as albedo (a measure of solar radiation
absorption), humidity, and temperature could vary within an orchard
\cite{landsberg1973}. The bulk of remaining research with respect to apple
orchard microclimates is how hailnets and shading screens affect crop yield and
quality of the fruit.


