\section{\emph{Internet of Things}}

\subsection{What is IoT?}

The Internet of Things (IoT) refers to the concept of integrating networking
capability in a range of devices, allowing for cooperation to reach common goals
\cite{atzori2010}. A similar but more colloquial term would be "smart" devices.

IoT has become increasingly common in recent years as a number of technological
breakthroughs has made deploying these networks more practical. These advances,
which are referred to as "enabling technologies", provide the foundation for
modern IoT systems. LoRa is a key example described in Chapter \ref{sec:lora}, a
further list of enabling technologies is available as a supplement in Appendix
\ref{app:enabling-tech}.

\subsection{The layers of IoT}

All IoT devices rely on the existence of three fundamental network layers
\cite{burhan2018iot} - working from the bottom to the top these are:

\begin{enumerate}
  \item Perception - This layer contains sensors that collect information about
        conditions in the world around them. This layer may perform some
        transformations on the data it receives (e.g. sorting, formatting) or just
        transmit raw data. An example of this is a smart car charger collecting data
        on the charge level of an electric car.
  \item Network - This layer acts as the bridge between the perception layer and
        the application layer. It is the medium and protocols associated with the
        transmission of data and the hardware necessary to interpret this. Following
        the above example this could be the use of WiFi or a mesh protocol such as
        Zigby to transmit the car's current charge level to a gateway - such as a WiFi
        router.
  \item Application - This encompasses any software that manipulates, displays
        or otherwise uses data from the perception layer. This could be hosted on the
        cloud or locally. In the example above this could be an app that shows the
        current charge status of the car.
\end{enumerate}

The hardware overview in Chapter \ref{sec:hardware-overview} presents the
components of my IoT solution in the context of these layers.

\subsection{Studies using LoRa IoT weather stations in agriculture}

The use of IoT in agriculture has become widespread in recent years as it offers
the opportunity for farmers to improve yields and cut costs by bringing digital
solutions that would not have previously been viable without access to power and
internet connectivity. The use of IoT in agriculture is often wrapped up in the
moniker of "Smart Farming", which encompasses a range of digital, robotic, and
internet-enabled approaches to improving efficiency. A 2019 review by Farooq et
al. \cite{farooq2019iot} reveals the scope of IoT applications in agriculture.
These include precision farming (IoT weather monitoring), automated irrigation,
pest and disease prediction with machine learning, and even the deployment of
agricultural drones for spraying, mapping and imaging.

There have been a few studies around the use of LoRa in smart farming
applications. In \cite{edgeAiGiaEtAl} LoRa was used in an edge computing
exercise - a procedure to compress data from multiple LoRa sensor nodes is
created and analysed. However, a criticism of this study is that they only
implement a single sensor and all results are drawn from a computer generated
sample. Additionally the range used for this single sensor to the gateway is
only 200m, which is not a realistic distance for most agricultural purposes
which the study is explicitly targeting.

The study in \cite{smartFarmKodaliEtAl} implements a LoRa based weather station
prototype in India. The authors create a fairly simple node with two sensor
types. After this the data from the nodes is viewed using an OLED screen on the
receiver. A significant criticism of this work is that it fails to specify the
range at which the LoRa communication was tested, which is a critical parameter
given that the technology was chosen specifically for its long-range
capabilities. Additionally, while the paper suggests the model is suitable for
large-scale deployment in a farm field , the experimental setup only consists of
a single transmitter node and gives no details on whether it had any weather
proofing.

This project aims to contribute to this literature by producing a more complete
weather station prototype that is weatherproofed for outdoor deployment using
multiple nodes and validated over more realistic distances.


\subsection{Commercial IoT agricultural solutions with LoRa}

While IoT weather station solutions are available commercially there are issues
with using these.

Virtually all affordable weather stations rely on WiFi to stay connected. This
is not viable for most farms as the fields they need to be deployed in are a
long way from buildings with WiFi access. As discussed in
section~\ref{sec:lora-benefits}, WiFi has just a fraction of the range of LoRa
so these types of stations are not comparable to the system I have developed.
This is explored in greater depth in the evaluation (see Chapter
\ref{hardware-evaluation}).
