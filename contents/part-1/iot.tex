\section{\emph{Internet of Things}}

\subsection{What is IoT?}

The Internet of Things (IoT) refers to the concept of integrating networking
capability in a range of devices, allowing for cooperation to reach common goals
\cite{atzori2010}. A similar but more colloquial term would be "smart" devices.

The number of IoT devices is forecasted to reach 40 billion by 2030 from 16.6
billion in late 2023 \cite{sinha2024}. IoT devices are already used extensively
in a multitude of commercial and domestic settings.

\subsection{The layers of IoT}

All IoT relies on the existence of three fundamental network layers
\cite{burhan2018iot}, working from the bottom to the top there is:

\begin{enumerate}
  \item Perception - This layer contains sensors that collect information about
  conditions in the world around them. This layer may perform some
  transformations on the data it receives (e.g. sorting, formatting) or just
  transmit raw data. An example of this is a smart car charger collecting data
  on the charge level of an electric car.
  \item Network - This layer acts as the bridge between the perception layer and
  the application layer. It is the medium and protocols associated with the
  transmission of data and the hardware necessary to interpret this. Following
  the above example this could be the use of WiFi or a mesh protocol such as
  Zigby to transmit the car's current charge level to a gateway - such as a WiFi
  router.
  \item Application - This encompasses any software that manipulates, displays
  or otherwise uses data from the perception layer. This could be hosted on the
  cloud or locally. In the example above this could be an app that shows the
  current charge status of the car.
\end{enumerate}

\subsection{IoT enabling technologies}

The surge in IoT for the past few decades has been fuelled by the emergence of
new technologies. Here I will explore the technologies that have are most
relevant to the weather stations built for this project.

\begin{enumerate}
  \item Efficiency improvements in microchips - breakthroughs in microchip
  fabrication have led to smaller more efficient chips with improved
  performance.
  \item Lithium-Ion batteries improvements - continuous improvements in the
  energy density of lithium-ion batteries has made it possible to power devices
  for long periods without mains power.
  \item Low-power long-range radio - new radio communication techniques such as
  LoRa allow for data transmission over several kilometres using a fraction of
  the power required by traditional mobile or Wi-Fi technologies.
  \item Affordability of solar panels - since 1970 the price of solar panels has
  decreased to 1/500th of its original cost \cite{economist2024} making solar a
  viable power source for IoT systems.
  \item Growth of hobbyist embedded systems - since the release of accessible
  platforms such as Arduino in 2005, the growth of hobby level embedded systems
  has lowered the barrier to entry to create IoT systems.
  \item Accessible cloud computing and hosting - Cheap and available web hosting
  has allowed application level systems to be more easily developed.
\end{enumerate}

\subsection{Studies using LoRa IoT weather stations in agriculture}

The use of IoT in agriculture has become widespread in recent years as it offers
the opportunity for farmers to improve yields and cut costs by bringing digital
solutions that would not have previously been viable without access to power and
internet connectivity. The use of IoT in agriculture is often wrapped up in the
moniker of "Smart Farming", which encompasses a range of digital, robotic, and
internet-enabled approaches to improving efficiency.

A 2019 review by Farooq et al. \cite{farooq2019iot} reveals the scope of IoT
applications in agriculture. These include precision farming (IoT weather
monitoring), automated irrigation, pest and disease prediction with machine
learning, and even the deployment of agricultural drones for spraying, mapping
and imaging. 

IoT weather stations like those developed in this project are a growing trend in
agriculture.

\subsection{Commercial IoT agricultural solutions with LoRa}

While IoT weather station solutions are available commercially there are issues
with using these. 

Virtually all affordable weather stations rely on WiFi to stay connected which
is not viable for most farms as there is a large separation between the fields
they need to be deployed in and buildings with WiFi access. As discussed in
section~\ref{sec:lora-benefits}, WiFi has just a fraction of the range of LoRa
so these types of stations are not comparable to my own.

Limiting the search for weather stations with LoRa then there are a handful of
commercial options. The issue with these is cost, the prices of most models
(Decentlab Eleven Parameter, Innon WTS506). For a comparison of different
models see the relevant section in the evaluation.

