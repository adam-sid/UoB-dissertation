\section{LoRa}

\subsection{What is LoRa?}

https://www.youtube.com/watch?v=jHWepP1ZWTk
https://www.youtube.com/watch?v=T3dGLqZrjIQ
- see further reading on second one

LoRa stands for \textbf{Lo}ng \textbf{Ra}nge and it is a relatively modern radio
modulation technique that allows for the transmission of data very long
distances. Another comparable modulation technique is WiFi where LoRa has 4000
times the range (REFERENCE), with a theoretical maximum of 800km (although far
less in realistic conditions). 

This long distance can be achieved with remarkably little power, WiFi typically
transmits data with 100mW-200mW of power for a range of 100-200m. LoRa meanwhile
uses only 25mW of power.

\subsection{How LoRa works}

The preamble of a LoRa packet sends the same symbol multiple times in a row to
allow the receiver to synchronise.

The most technically impressive aspect of LoRa is that it a receiver can
demodulate signals below the noise floor. The noise floor is the sum of all
unwanted signals in an environment. To use an example a stadium full of people
talking will create a certain level of background noise, in order to speak to
someone else in the stadium you would have to speak at a volume which can
overcome this background noise. The fact that LoRa receivers can successfully
demodulate below the noise floor is akin to being able to whisper in such a
stadium and still be understood.

\subsection{Benefits and limitations}

Fresnell zone


\subsection{Current applications of LoRa}