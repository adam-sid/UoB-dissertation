\section{LoRa}\label{sec:lora}

Before analysing IoT systems more generally, I will explain the most critical
hardware enabling technology for this project, which is the radio communication
technique known as LoRa.

\subsection{What is LoRa?}

LoRa stands for \textbf{Lo}ng \textbf{Ra}nge and it is a radio modulation
technique invented in 2014 that allows for the transmission of data over very
long distances. It is one of several competing low power wide area networks
(LPWAN) but the hardware to use it is much more available than these other
networks. LoRa has a range over 4000 times greater than WiFi \cite{spiess2019},
a range that it can achieve with remarkably little power (Table
\ref{tab:lora-stats}). This makes LoRa the preferred technology choice for the
remote, off-grid application in this project.

For a brief overview of the principles behind LoRa I have written the
supplementary section "How LoRa works" in Appendix \ref{app:lora-explained}.
However, to summarise this, the most important principle that separates LoRa
from traditional radio modulation is the fact that it can demodulate incoming
signals more efficiently.

\subsection{Benefits and limitations}\label{sec:lora-benefits}

The benefit from the ability of the LoRa receiver to demodulate signals more
efficiently is two-fold. First it reduces the power needs on transmitter and
receiver: the transmitter sends less powerful signals because the receiver can
more easily distinguish signals; in turn the receiver can demodulate with a
lower power budget because of the easier correlation process with LoRa chirps
(Figure \ref{fig:lora-wave} from supplementary Appendix).

The second related benefit is the ability for the receiver to demodulate signals
which are below the noise floor (the sum of all interfering signals).
Essentially, even when background noise is 'louder' than the LoRa signal, the
receiver is still able to distinguish and process the data in the signal. This
helps LoRa transmitters to broadcast signals with a far greater effective range
despite its low power. Table \ref{tab:lora-stats} summarises LoRa against other
well known wireless communication technologies, showing the relative advantages
or LoRa. The main disadvantage of LoRa is that it has a comparatively lower data
throughput, however the packets involved in this project were only around 20-30
bytes so this was not a concern.

\begin{table}[ht]
  \centering
  \begin{tabular}{|l|l|p{4.5cm}|r|}
    \hline
    \textbf{Technology} & \textbf{Wireless Communication} & \textbf{Range} &
    \textbf{Tx Power} \\
    \hline
    Bluetooth           & Short range                     & 10 m & 2.5 mW  \\
    \hline
    WiFi                & Short range                     & 50 m & 80 mW   \\
    \hline
    3G/4G               & Mobile network                  & 5,000m & 5000 mW \\
    \hline
    LoRa                & LPWAN                           &
    \parbox[t]{4.5cm}{2,000--15,000m} & 20 mW   \\
    \hline
  \end{tabular}
  \caption{Comparison of wireless technologies (Source:
    \cite{lie_lora_readthedocs})}
  \label{tab:lora-stats}
\end{table}
