\section{Introduction}

In this report, I describe an online tool called \myReportTitle{}, which
collects and displays microclimate data from an IoT weather sensor network that
I designed and built. The tool aims to enable microclimate specific forecasting
and provide farmers with accessible, location specific climate data to support
decision making and enable microclimate specific forecasting. The IoT hardware
was built using off-the-shelf components and housed in hand made enclosures. The
deployed system consisted of four components: two field sensor nodes capable of
reading weather data and transmitting it using a modern radio protocol called
LoRa; a repeater that boosts received LoRa signals; and an internet connected
gateway that uploads the weather data to an online database. 

The weather station was designed for deployment on Small Brook Farm in Devon,
unfortunately actual deployment at this location was not possible during the
time frame so the evaluation section of this report is focussed on data
collected while the weather station was set up in my garden. At the time of
submission, discussions with the owner of Small Brook Farm were ongoing to
deploy the system in their orchard as part of a funded university project known
as DECIDE.

\subsection{Motivation}

Weather forecasts are typically based on data from distant weather stations and
large scale models which fail to capture variations that exist within a single
farm or even field. These local variations, known as microclimates, can differ
significantly even across relatively short distances due to differences in
factors such as elevation, tree cover, or soil conditions. The weather
conditions on a farm 

One major source of motivation for this project comes a set of interviews with the owners of Small Brook Farm (see relevant excerpt in ~\ref{sec:small-brook-interview} )

See excerpt in 



\subsection{Aims and contributions}

The main aim of this project was to develop a low-cost, low-maintenance IoT
solution that helps farmers monitor microclimate conditions and make better
informed decisions. The final hardware system needed to be physically robust to
survive outside while the the software needed to present relevant information in
an intuitive way, ensuring that users can easily make use of the data.

The key contributions of this paper are:

\begin{itemize}
    \item The development of a low cost, ultra long-range remote weather station
    system with superior range and reduced cost compared to current solutions on
    the market.
    \item A publicly available web application that visualises live data from
    multiple field locations.
    \item A method for enabling microclimate forecasting by comparing local
    sensor data with broader regional weather forecasts.
\end{itemize}

\subsection{Layout of dissertation}

This dissertation has four parts. Part 1 provides the technical background to
understand the project, including an overview of the core technologies used and
a review of related work in the field. Part 2 details the hardware development
process, covering the rationale behind component selection, the testing of
hardware, and how the IoT nodes were deployed in the field. Part 3 focuses on
the software aspect of the project, outlining the design and implementation of
both the backend infrastructure and the frontend web interface. Part 4 then
offers a critical evaluation of the system, discussing its performance,
usability, and limitations, as well as highlighting opportunities for future
development improvements and research.

