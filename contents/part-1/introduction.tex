\section{Introduction}

In this report, I describe an online webapp called \myReportTitle{}, which
collects and displays microclimate data from an \emph{Internet of things} (IoT)
weather station network that I designed and built.The webapp can present live,
historical and forecasted weather data to the user. The live and historical data
is retrieved from a separate postgreSQL database that collects and stores
readings from my weather station sensor nodes every minute. These historical
readings were used to build a machine learning model that can predict future
microclimate specific weather based on the general forecast.

The weather station network was built using off-the-shelf components and housed
in hand made enclosures. The deployed system consisted of four components: two
field sensor nodes capable of reading weather data and transmitting it using a
modern radio protocol called LoRa; a repeater that boosts received LoRa signals;
and an internet connected gateway that uploads the weather data to an online
database. The hardware in this project was designed for deployment on Small
Brook Farm, an apple orchard who primarily sell their produce to cider makers.
Unfortunately, actual deployment at this location was not possible during the
time frame so the evaluation section of this report is focused on data collected
while the weather station network was set up in my garden. 

At the time of submission, discussions with the owner of Small Brook Farm were
ongoing to deploy the system in their orchard as part of a funded university
project known as DECIDE.

\subsection{Motivation and aims}

Weather forecasts are typically based on data from distant weather stations and
large scale models which fail to capture variations that exist within a single
farm or even field. These local variations, known as microclimates, can differ
significantly even across relatively short distances due to differences in
factors such as elevation, tree cover, or soil type. 

For example, in a set of interviews with the owners of Small Brook Farm
(Appendix ~\ref{sec:small-brook-interview}) the owners note that the use of
traditional forecasts on their orchard is not particularly useful for them. As
weather forecasts tend to cover wide areas and are not always indicative of very
local conditions. The owners even mentioned that the weather conditions across a
single field of apple trees was different, with higher winds on one side
compared to other. 

This uncertainty has real world consequences for farmers, at small brook farm
for instance wind speed is a key determinant of when the farmers can spray their
crops with pesticides. Equally, general forecasts are not well equipped to
predict a spring frost that damages crops or the precise incubation period of
pests and diseases.

Agriscanner aims to address this need for accessible local weather forecasting
by giving farmers a weather station platform that can both relay current weather
information from different sections of their field and predict future weather
for their specific location. 

\subsection{Contributions}

The key contributions of this paper are:

\begin{itemize}
    \item The development of a low cost, ultra long-range remote weather station
    system with superior range and reduced cost compared to current solutions on
    the market.
    \item A publicly available web application that visualises live data from
    multiple field locations, with a high degree of usability when evaluated
    qualitatively using the System Usability Scale (SUS).
    \item A novel method for enabling microclimate forecasting using a machine
    learning process that compares local sensor data with broader regional
    weather forecasts.
\end{itemize}

\subsection{Layout of dissertation}

Part 1 provides the technical background to understand the project, including an
overview of the core technologies used and a review of related work in the
field.

Part 2 details the hardware development process, covering the rationale behind
component selection, the testing of hardware, and how the IoT nodes were
deployed in the field.

Part 3 focuses on the software aspect of the project, outlining the design and
implementation of both the backend infrastructure and the frontend web
interface.

Part 4 then offers a critical evaluation of the system, discussing its
performance, usability, and limitations, as well as highlighting opportunities
for future development improvements and research. 


  

