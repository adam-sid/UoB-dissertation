\section{Hardware evaluation}

\subsection{Qualitative discussion of weather station performance}

\subsubsection{Battery life, solar power and outages}

The nodes have had significantly more downtime than expected based on the energy
budgets in Sections~\ref{sec:battery-tests} and~\ref{sec:solar-tests}, despite
doubling the battery capacity during deployment. The primary cause is site
placement, with both nodes in a north-facing garden bounded by 2.5\,m fencing
and a two-storey house immediately to the south. As a result, they receive no
direct sunlight until 09:00, and Node 1 is shaded again by 15:00 due to the
shadow of a nearby garage. This leaves only \(\approx\)5\,h of direct sunlight,
which is far below the amount assumed in the design of the nodes.

Outages seem to only affect the sensor nodes; the repeater has had zero downtime
since installation. This is consistent with its much lower energy budget from a
lack of any peripheral sensors, which reduces average energy draw.

The limitations of the current location, plausibly explain the observed outages.
Because these shading conditions are unlikely in an open field, these outages
are not of primary concern for the intended deployment. However, it highlights
the system's sensitivity to late sunrise and early sunsets from terrain. For
example if the actual deploy site was shaded by a hill or some tall trees late
in the day, which I had not properly considered in the design phase.

\subsubsection{Weatherproofing}

The summer of 2025 is set to be one of the hottest and driest on record with
August specifically receiving roughly half its average rainfall
\cite{uor2025summer}. Consequently, from the 15th of August to the 28th of
August the weather stations saw only short spells of rain, making an analysis of
their weather proofing hard to review up to this point. 

Fortunately, from the 28th August to the 2nd September Chipping Sodbury - where
the nodes are currently placed - received roughly 3.4cm of rain according to
weather forecasts. Throughout this period there has been no evidence of water
ingress.

Beyond ingress testing, the nodes experienced elevated ambient temperatures
(exceeding \SI{32}{\degreeCelsius}) and showed no signs of overheating. These
results are encouraging, although longer-term deployment that includes colder
weather and sustained heavy rain is needed to comment conclusively on their
overall robustness.

\subsubsection{Range}

As reported in section ~\ref{sec:range-tests} the challengers and antennas used
in this configuration are able to reach at least 1200m of range in non-ideal
circumstances.

\subsubsection{Reset behaviour}

As covered briefly in Chapter~\ref{sec:deployment}, the reset behaviour of the
challenger is inconsistent after a brownout or complete power outage.
Essentially if the power of the sensor nodes is cut off at night then it is not
guaranteed that the node will restart the code.py program in the morning. This
is in spite of the fact that I modified the safemode.py behaviour of
circuitpython with the recommended code for remote solar power applications. As
this is ultimately appears to be a firmware issue there is not an obvious or
easy solution for this. 

With enough time I would probably rewrite all of the challenger software in
Micropython and see if that firmware behaves in a more useful way on power
outage. Unfortunately the issue was picked up fairly late in development and it
was unrealistic to perform a refactor of this scale with the time left.

\subsection{Quantitative comparison with commercial alternatives}

Section will compare my node to others in a table and conclude on strengths and
weaknesses of each
