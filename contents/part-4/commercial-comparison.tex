\section{Hardware evaluation}

\subsection{Qualitative discussion of weather station performance}

\subsubsection{Battery life, solar power and outages}

The nodes have had significantly more downtime than expected based on the energy
budgets in Sections~\ref{sec:battery-tests} and~\ref{sec:solar-tests}, despite
doubling the battery capacity during deployment. The primary cause is site
placement, with both nodes in a north-facing garden bounded by 2.5\,m fencing
and a two-storey house immediately to the south. As a result, they receive no
direct sunlight until 09:00, and Node 1 is shaded again by 15:00 due to the
shadow of a nearby garage. This leaves only \(\approx\)5\,h of direct sunlight,
which is far below the amount assumed in the design of the nodes.

Outages seem to only affect the sensor nodes; the repeater has had zero downtime
since installation. This is consistent with its much lower energy budget from a
lack of any peripheral sensors, which reduces average energy draw.

The limitations of the current location, plausibly explain the observed outages.
Because these shading conditions are unlikely in an open field, these outages
are not of primary concern for the intended deployment. However, it highlights
the system's sensitivity to late sunrise and early sunsets from terrain. For
example if the actual deploy site was shaded by a hill or some tall trees late
in the day, which I had not properly considered in the design phase.

\subsubsection{Weatherproofing}

The summer of 2025 is set to be one of the hottest and driest on record with
August specifically receiving roughly half its average rainfall
\cite{uor2025summer}. Consequently, from the 15th of August to the 28th of
August the weather stations saw only short spells of rain, making an analysis of
their weather proofing hard to review up to this point. 

Fortunately, from the 28th August to the 2nd September Chipping Sodbury - where
the nodes are currently placed - received roughly 3.4cm of rain according to
weather forecasts. Throughout this period there has been no evidence of water
ingress.

Beyond ingress testing, the nodes experienced elevated ambient temperatures
(exceeding \SI{32}{\degreeCelsius}) and showed no signs of overheating. These
results are encouraging, although longer-term deployment that includes colder
weather and sustained heavy rain is needed to comment conclusively on their
overall robustness.

\subsubsection{Range}\label{sec:range-eval}

As reported in section ~\ref{sec:range-tests} the challenger microcontrollers
and antennas used in this configuration are able to reach at least 1200m of
range in a real world situation. The conditions in that test were not ideal for
LoRa as both the receiver and transmitter were positioned low to the ground at
around 1-1.5m, additionally line of sight was broken at around 1000m which
severely diminishes the performance of any radio communication system. There
were also buildings between 1200m and 1400m that blocked LoRa signals entirely.

In the intended deployment of my full weather network, the use of an additional
repeater would significantly improve range. In a flat terrain setting then the
use of a repeater would clearly double the range, however if terrain is hilly a
repeater can increase range by even larger multiples. The below graphic
demonstrates the problem of hills when it comes to LoRa. As a LoRa signal
propagates from the transmitter and collides with a hill the majority of the
signals energy is reflected off the hill meaning it cannot reach the gateway.

\begin{figure}[H]
    \centering
    \includegraphics[width=0.9\textwidth]{contents/part-4/fig4/no-repeater.png}
    \caption{Illustration of LoRa radio propagation between node and receiver with blocking terrain}
    \label{fig:no-repeater}
\end{figure}

If a repeater is placed on the top of the hill not only would the LoRa signal
reach the gateway receive signal but the repeaters increased elevation would
give a significant boost to its range as the total area with a line of sight to
the repeater would be vastly increased.

The range that this repeater allows is what sets the design of my network apart
from commercial options and is discussed in more detail in the next section..

\begin{figure}[H]
    \centering
    \includegraphics[width=0.9\textwidth]{contents/part-4/fig4/repeater.png}
    \caption{Illustration of LoRa radio propagation with a repeater included}
    \label{fig:repeater}
\end{figure}

\subsubsection{Reset behaviour}\label{sec:reset-behaviour}

As covered briefly in Chapter~\ref{sec:deployment}, the reset behaviour of the
challenger is inconsistent after a brownout or complete power outage.
Essentially if the power of the sensor nodes is cut off at night then it is not
guaranteed that the node will restart the code.py program in the morning. This
is in spite of the fact that I modified the safemode.py behaviour of
circuitpython with the recommended code for remote solar power applications. As
this ultimately appears to be a firmware issue there is not an obvious or easy
solution for this. 

With enough time I would probably rewrite all of the challenger software in
Micropython and see if that firmware behaves in a more useful way on power
outage. Unfortunately the issue was picked up fairly late in development and it
was unrealistic to perform a refactor of this scale with the time left.

\subsection{Quantitative comparison with commercial alternatives}

I will now compare the performance of my LoRa weather station to prebuilt
options available to purchase. A breakdown of costs for Agriscanner is
included in Appendix~\ref{app:cost-nodes}.

\begin{table}[H]
\centering
\small
\renewcommand{\arraystretch}{1.2}
\begin{tabularx}{\textwidth}{l >{\raggedright\arraybackslash}X
  >{\raggedright\arraybackslash}X >{\raggedright\arraybackslash}X
  >{\raggedright\arraybackslash}X >{\raggedright\arraybackslash}X}
\hline
 & \textbf{Agriscanner Network} & \textbf{SenseCAP
 S2120\cite{pihut:sensecap-s2120-2025}} & \textbf{Decentlab Eleven
 Parameter\cite{alliot:decentlab-eleven-2025}} & \textbf{HOBO weather station
 kit\cite{weathershop:hobo-rx3000-2025}} & \textbf{SparkFun Arduino weather
 kit\cite{pihut:sparkfun-2025}} \\

\hline
Number of sensors & 4 & 8 & 11\textsuperscript{*} & 6 & 7 \\
Sensor accuracy & Hobbyist & Hobbyist & Professional & Professional & Hobbyist
\\
Communication type & LoRa & LoRa & LoRa & Mobile network & WiFi \\
Update frequency & 1 minute & 1 hour & 10 minutes & 1 hour & 1 minute \\
Readings per hour & 60 & 1 & 6 & 10 & 60 \\
Power source included & Yes & Yes & Yes & Yes & No \\
Power source & Solar & Solar & Solar & Solar & -- \\
Batteries recharge? & Yes & No & No & Yes & -- \\
Reported battery life & replace $\sim$ 3 years & 154 days & Several months &
replace 3--5 years & -- \\
Reported range & -- & 2--10\,km & 2--10\,km & Anywhere with 4G & -- \\
Estimated range & 2.4--20\,km & 1.2--10\,km & 1.2--10\,km & -- & 10--50\,m \\
IP rating & $\sim$ IP65 & IPX6 & IP66 & IP66 & None \\
Ongoing payment? & -- & -- & -- & Yes -- mobile plan & -- \\
Ongoing costs p.a & \pounds{}0 & \pounds{}0 & \pounds{}0 & \pounds{}132 &
\pounds{}0 \\
Cost per sensor node & \pounds{}177 & \pounds{}287 & \pounds{}3{,}272 &
\pounds{}4{,}138 & \pounds{}130 \\
Cost per repeater & \pounds{}93 & \pounds{}0 & \pounds{}0 & \pounds{}0 &
\pounds{}0 \\
Cost per gateway\textsuperscript{**} & \pounds{}66 & \pounds{}122 & \pounds{}122
& \pounds{}0 & \pounds{}0 \\
Battery cost p.a.\textsuperscript{***} & \pounds{}7 & \pounds{}10 & \pounds{}30
& \pounds{}20 & \pounds{}0 \\
\textbf{Total cost\textsuperscript{****}}  & \textbf{\pounds{}520} &
\textbf{\pounds{}706} & \textbf{\pounds{}6{,}696} & \textbf{\pounds{}8{,}418} &
\textbf{\pounds{}260} \\
\hline
\end{tabularx}

\vspace{0.25em}
\textit{\footnotesize * Sensors missing from Agriscanner: Solar radiation,
rainfall, barometric pressure, vapor pressure, dew point, wind direction, tilt
sensor, lightning strike count / distance} \\
\textit{\footnotesize ** For SenseCAP and Decentlab models the lowest cost
gateway available is sensecap m2 at £122} \\
\textit{\footnotesize *** Battery cost assumptions are detailed in
Appendix~\ref{app:battery-assumptions}.} \\
\textit{\footnotesize **** Includes sensors, repeater, gateway, and estimated first-year battery + ongoing costs.}
\caption{Comparison of weather-station options.}
\label{tab:commercial-comparison}
\end{table}

\subsubsection{Benefits of my weather station}

\begin{itemize}
  \item \textbf{Cost:} At £520 the Agriscanner weather stations are the lowest
  cost option among waterproof, long-range systems. While the SparkFun kit is
  cheaper, it is not suitable for outdoor deployment due to its exposed
  electronics and, since it relies on WiFi, its range would be insufficient in
  an agricultural setting. It will therefore not be discussed further. The two
  professional systems (DecentLab and HOBO) are over ten times the price of my
  system and are therefore not comparable.
  
  The only system with similar characteristics under £1,000 that I could find is
  the SenseCAP device. However, with a cost per node roughly 50\% higher, my
  nodes still provide better value. If additional nodes were deployed, this
  price difference would become even more significant.

  \item \textbf{Frequency of readings:} My sensor nodes collect and transmit
  readings every minute, providing the Agriscanner webapp with effectively live
  data. By contrast, most alternative options only provide hourly readings,
  leaving actual conditions between measurements unknown.

  \item \textbf{Battery recharging:} My nodes use rechargeable 18650 batteries,
  which I conservatively estimate will last around three years (the same battery
  chemistry as mobile phones) before requiring replacement. Of the alternatives,
  only the HOBO system features a rechargeable battery. Both SenseCAP and
  DecentLab rely on disposable alkaline batteries, requiring regular replacement
  and therefore more frequent maintenance. 

  \item \textbf{Range:} As explained in Section~\ref{sec:range-eval}, my weather
  station network benefits from the inclusion of a repeater. This enables
  significantly greater range than the other models could provide, particularly
  in hilly terrain.
  
  I also doubt whether either the SenseCAP or DecentLab systems could achieve a
  substantial range improvement under the same test conditions. In the UK, LoRa
  transmission power is subject to a strict regulatory cap (see
  Section~\ref{sec:lora-limit}). Since my system already operates at this
  maximum power, commercial products cannot exceed it. Even with a superior
  antenna, transmit power would need to be reduced proportionally, as antenna
  gain also counts towards the effective radiated power. It is therefore highly
  unlikely that these alternatives would achieve a range advantage sufficient to
  offset the benefits of a repeater in my system.
\end{itemize}

\subsubsection{Drawbacks of my weather station}

\begin{itemize}
  \item \textbf{Number of sensors:} One of the clearest drawbacks of the
  Agriscanner system is the lower sensor count: only four sensors compared with
  seven to eleven on competing systems. However, the Challenger microcontrollers
  in my sensor nodes still have spare GPIO ports, so adding additional sensors
  would be straightforward. The cost of many sensors is also not prohibitive
  (for example, a UV index sensor is only \pounds{}6).
  \item \textbf{Sensor accuracy:} Professional systems use higher-grade sensors
  with tighter accuracy specs. For example, the DecentLab unit specifies air
  temperature accuracy of $\pm0.6\,\si{\celsius}$, whereas the DHT11 used in
  mine is only rated to about $\pm2.0\,\si{\celsius}$.
  \item \textbf{Reset behaviour:} As noted in Section~\ref{sec:reset-behaviour},
  the node firmware can behave unreliably after a full battery drain. While I
  cannot purchase the other devices in this list to confirm this directly, it is
  unlikely that the other units would exhibit the same behaviour, especially
  because their firmware will built specifically for remote solar powered
  applications.
  \item \textbf{Other factors:} The other devices bring other non-tangible
  benefits besides their technical specification. For example,  warranties,
  formal testing, vendor support etc. Professional systems commonly offer
  long-term maintenance and firmware updates beyond the initial purchase, which
  my prototype solution obviously does not offer. 
\end{itemize}

\subsection{Conclusion on hardware performance}

While the weather station network has shown promising results as a low cost and
long range solution, the unreliable reset behaviour holds it back from being
100\% deployment ready. So far the weatherproofing of the external devices has
been encouraging, though testing has been limited due to unusually dry weather
and a longer deployment in wetter and colder conditions is needed. The outages
seen during the current deployment appear to be an issue specific to the private
garden they are in and would be much less likely in an open field environment;
however, this does emphasise the need for more extensive site surveys prior to
installation. Overall, if the reset behaviour can be addressed and a small
number of additional sensors are added, the complete system would be well suited
for long term unattended deployment on a farm.