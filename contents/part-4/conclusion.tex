\section{Future work}\label{sec:future-work}

Ultimately the main objective of this work is to deploy the system in an apple
farm to assist the farmers with the management of their orchard. Before the
system can be deployed, the Challenger's firmware must be able to reliably
restart after a power outage.  As suggested in section \ref{sec:reset-behaviour}
I would first recommend experimenting with MicroPython to see if this can
provide a more robust solution to coping with this scenario. The worst case
scenario would be to have to use an alternative embedded processor that can
handle power fluctuations.

Additionally, it may be of value to add more sensor to the nodes as this is a
noticeable weakness of the current system architecture compared to existing
alternatives. Considering the addition of sensors would result in increased
power requirements I believe it would be inadvisable to add any more for time
being. Once the system has been deployed for an extended period and shows that
it can operate with minimal outages then the adding sensors would be a
worthwhile improvement to the current design.

Once the reliability issue has been resolved, the next stage will be to deploy
the system onto a farm.  The siting of the sensor nodes and repeater will need
careful consideration, so a thorough site survey should be undertaken prior to
this. It would be advisable to allow a few days for the installation, so there
is someone on hand to deal with any issues that might arise once the nodes have
been installed and to make sure the farmers are comfortable with using the
webapp.

I would recommend that the forecasting feature continues to be developed using
machine learning. Furthermore instead of training the model offline and manually
inserting models into the webapp, future work could focus on moving model
training online. This would allow the model to continually update and improve as
new data from the nodes is produced. After a few months of data is collected
then a repeated comparison should be performed to 


\section{Closing remarks}

This dissertation has shown the process of building a hardware based weather
station network from the ground up and integrating this into a modern and usable
web application. 

On the hardware side a weather station network has been successfully built and
deployed externally. Despite heavy rain the devices have continued to operate
and require no mains power supply relying soley on the sun.