\section{Future work}\label{sec:future-work}

Ultimately the main objective of this work is to deploy the system in an apple
farm to assist farmers with the management of their orchard. Before the system
can be deployed, the Challenger's firmware must be able to reliably restart
after a power outage.  As suggested in section \ref{sec:reset-behaviour} I would
first recommend experimenting with MicroPython to see if this can provide a more
robust solution to coping with this scenario. The worst case scenario would be
to have to use an alternative embedded processor that can handle power
fluctuations.

Additionally, down the line it may be of value to add more sensor to the nodes
as this is a noticeable weakness of the current system architecture compared to
existing alternatives. Considering the addition of sensors would result in
increased power requirements I believe it would be inadvisable to add any more
for time being. Once the system has been deployed for an extended period and
shows that it can operate with minimal outages then the adding sensors would be
a worthwhile improvement to the current design.

Once the reset behaviour issue has been resolved, the next stage will be to
deploy the system onto a farm.  The siting of the sensor nodes and repeater will
need careful consideration, so a thorough site survey should be undertaken prior
to this. It would be advisable to allow a few days for the installation, so
there is someone on hand to deal with any issues that might arise once the nodes
have been installed and to make sure the farmers are comfortable with using the
webapp.

I would recommend that the forecasting feature continues to be developed using
machine learning. Furthermore instead of training the model offline and manually
inserting models into the webapp, future work could focus on moving model
training online. This would allow the model to continually update and improve as
new data from the nodes is produced. After a few months of data is collected
then the evaluation in Chapter \ref{sec:mach-learn-eval} should be performed
again to analyse if the prediction quality has improved.


\section{Closing remarks}

This dissertation has shown the process of developing a complete IoT weather
station network with an accompanying webapp. In the hardware evaluation it was
shown that the system designed here offers superior range at lower cost compared
to commercial options. The system has continued operating despite rainfall and
high temperatures, suggesting the design here is relatively well weather proofed.

The webapp designed here scored well in terms of usability, achieving a far
higher SUS score than the benchmark level of 68. The user stories developed from
survey responses also offer a clear roadmap to future development.

Finally the machine learning evaluation explored the accuracy of the current
machine learning system against a simpler model. With additional time in the
field and training data, it is expected that the models accuracy will see large
improvements.