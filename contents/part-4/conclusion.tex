\section{Future work}

Ultimately the objective of this work is to deploy the system in an apple farm
to assist the farmers with the management of their orchards. Before the system
can be deployed, the Challenger's firmware must be able to reliably restart
after a power outage.  As suggested in section \ref{sec:reset-behaviour} I would
first recommend experimenting with MicroPython to see if this can provide a more
robust solution to coping with this scenario. The worst case scenario would be
to have to use an alternative embedded processor that can handle power
fluctuations.

Additionally, it may be of value to add more sensor to the nodes as this is a
noticeable weakness of the current system architecture compared to existing alternatives.
Additional sensors would result in increased power requirements, and I believe
it would be inadvisable to add any more load to the system until its performance
in the target setting can be assessed.

Once the reliability issue has been resolved, the next stage would be to deploy
the system into a farm.  The siting of the sensor nodes and repeater will need
careful consideration so a thorough site survey should be undertaken. It would
be advisable to allow a few days for the installation, so there is someone on
hand to deal with any issues that might arise once the nodes have been installed
and to make sure the farmers are comfortable with using the webapp.

I would recommend that the forecasting feature is turned off, or heavily
caveated initially as the predictive models need to be retrained with data from
the new site over a good number of weeks.  Once a more substantial data set is
available, the machine learning training process will need to be repeated and
deployed.  It will then be possible to repeat the analysis of how the predictive
models perform when trained on more representative data.


\section{Closing remarks}

Brief conclusion