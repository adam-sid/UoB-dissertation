\section{Related work}

\subsection{Studies of IoT in agriculture}

The use of IoT in agriculture has become widespread in recent years as it offers
the opportunity for farmers to improve yields and cut costs by bringing digital
solutions that would not have previously been viable without access to power and
internet connectivity. The use of IoT in agriculture is often wrapped up in the
moniker of "Smart Farming", which encompasses a range of digital, robotic, and
internet-enabled approaches to improving efficiency.

A 2019 review by Farooq et al.\ \cite{farooq2019iot} reveals the scope of IoT
applications in agriculture. These include precision farming (soil quality,
moisture, and weather monitoring), automated irrigation, pest and disease
monitoring and prediction with machine learning, and even the deployment of
agricultural drones for spraying, mapping and imaging. 

\subsection{Microclimate prediction using machine learning}

There have been a number of recent studies where machine learning has been used
to help predict micro climate conditions. A 2021 study by Kumar et al developed
a machine learning framework that is able to predict a variety of climatic
variables such as soil moisture, wind speed and temperature. They were able to
get up to 90\% accuracy with a 12-120 hour forecast range.

\subsection{IoT weather stations on the market}

