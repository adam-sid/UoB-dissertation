\section{\emph{Internet of Things}}

\subsection{LoRa and LoRaWAN}

Long range systems are a way of transmitting data over much longer distances than
Wi-Fi allows. LoRa has a range 4000 times greater than Wi-Fi.

\subsection{Implementation of LoRa Communication}

In this project, I deployed a low-power, long-range (LoRa) wireless network to
connect a distributed set of environmental sensors across an apple orchard. The
primary goal was to establish a reliable Internet-of-Things (IoT) weather
station capable of transmitting temperature, humidity, soil moisture, and wind
speed data over distances of up to [insert number of km] without the need for 
external infrastructure such as with mobile data.

Measurements were sent as compact JSON payloads. We achieved a balance of data
rate (insert data rate) and sensitivity (insert sensitivity), ensuring packet delivery
even over hilly terrain. 

On the gateway side, a [insert final device used] served as a base station. It
received raw payloads and forwarded them over MQTT to a cloud server for storage
and visualization. Field tests were conducted showing a maximum range of 
[final range achieved].

\subsection{Early prototyping}

For early prototyping, We chose two Challenger RP2040 LoRa modules. These are
simple Raspberry Pico based devices with a low powered antenna. With these
modules I developed an early script that allowed for the sending of JSON packets
containing all the necessary measurements. 

These modules served only as a proof of concept as my own testing of range for
these only reported about 100m distance before significant packet loss and
corruption occurred.