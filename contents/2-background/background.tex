\section{Background}

\subsection{Microclimates: definition and relevance}

A microclimate is generally understood as a set of distinct climatic conditions
that are distinct to a small, localised area \cite{MetOffice2023}. The maximum
size of a microclimate is debated, but the World Meteorological Organisation
(WMO) regards it as occupying an area of anywhere from less than one metre
across to several hundred meters \cite{wmo2024}.  In practice, microclimates can
occur in spaces such as gardens, valleys, caves, or fields. Even human-made
structures can generate their own microclimates; for example, tall buildings can
create \emph{street valleys} that reduce wind flow and lead to the formation of
localized pockets of warmer air, which can also trap higher concentrations of
pollution from vehicle emissions \cite{yang2023}. Vegetation plays a critical
role in influencing microclimates. The addition of trees to an urban environment
can reduce air temperature by as much as \SI{2.8}{\degreeCelsius}
\cite{lai2019}.

\subsection{Microclimates in apple orchards}

\subsection{Smart farming}

\subsection{\emph{Internet of things} and sensor networks}

