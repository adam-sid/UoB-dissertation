\section{Backend: Database and SQL API}

For the database, I chose Render as a good low cost option for hosting the web
service as it offers a free trial period, constant uptime and backups. The free
version of web services by Render (render.com), provides a limited set of
resources at zero cost: 256 MB of RAM; 0.1 share of a CPU and 1 GB of storage.
There are also limits on outbound bandwidth, build pipeline minutes and instance
hours. A single instance of a POSTGRES database is available on this service -
which is free for the first 30 days and then costs \$5 a month. These resources
should be sufficient for this project as only small amounts of data are
involved.

The web service is running an API I have written, which will receive data from
the raspberry pi used as the gateway. The data is sent using a POST request
with a JSON body which is parsed, processed and added to the database by the
API. Data can be retrieved for display by the frontend using a GET request.

The database design is very simple with a single table holding all the data so
the choice of database management system is not crucial. Having said that,
POSTGRES is a well established and reliable relational database management
system that complies with ANSI SQL standards and most importantly for this
project is supported by Render.

The 1 GB storage limitation allows for (x) rows of data with the current
database design.

This solution should scale well. Additional tables can be added to the POSTGRES
data base for additional sites. And in the unlikely event that the volumes
exceed the Render limits it is easy to pay more for additional resources.

\section{Frontend: Agriscanner webapp}

Overview of front-end