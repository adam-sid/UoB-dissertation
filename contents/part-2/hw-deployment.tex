\section{Deployment}

\subsection{Test deployment at home}

Before setting up the nodes at the farm it was important to run the equipment in
an area where repairs and updates could be more easily performed. The nodes were
therefore placed at different locations in my garden and ran continuously for a
period of (ADD PERIOD).

\subsubsection{Challenges and solutions from test deployment}

One of the first challenges from the test deployment was the fact the batteries
on the nodes were not quite sufficient to allow 24/7 readings. On the fourth day
of the run, after an overcast evening the night before, one of the nodes stopped
transmitting at around 6am - just before sunrise. To remedy this another battery
was fitted to each of the end nodes (but not the repeater as this had much lower
battery consumption from the lack of any sensors). This doubling of battery
capacity would make loss of power during the night much less likely. The fact
the solar panels could charge the battery fully by mid morning pointed to the
fact that solar capacity was sufficient but battery capacity was not.

A related problem was that when the node died it did not start repeating again
after this despite solar power returning and the battery getting recharged. I
discovered that when the microcontroller detected brownout - being a sudden dip
in voltage - the device would enter a safeboot mode. In this safeboot mode the
default code.py file would not automatically be run and instead the device would
create and use a safemode.py file that without modification would essentially
run infinitely unless the reset button on the device was pressed or the device
was powered off and on again.

Clearly this behaviour was not appropriate for field deployment where brownouts
would potentially occur whenever the battery was discharged completely. Even
with the additional battery capacity provided by a second battery there would
still be a high chance of this occurring on particularly dark days during
winter, where continuous uptime could not be guaranteed.

To remedy this, I modified the safemode.py file using a template I found in the
circuitpython documentation. This version of the safemode.py file was made
specifically for the case I was using i.e. remote solar powered projects where
manual reset of the board could not be achieved easily. Instead of entering an
infinite loop this safemode would (DESCRIPTION OF SAFEMODE).

A seperate issue was the behaviour of the temperature-humidity sensor when
exposed to direct sunlight. While the junction box used to encase the sensor was
painted white to reduce any heat transfer from solar radiation, the readings
taken on very sunny days showed a large delta of around 5 celsius between the
sensor readings and local air temperature readings that could not be explained
with the existence of a microclimate.

To reduce this effect an additional solar shield was made to surround the
sensor. Consisting a wooden frame and aluminium shield to block light falling on
the junction box. While this method did help to reduce the delta at midday to
around 3 celsius there was still clearly quite a pronounced effect. So to
further reduce any warming the solar panel was mounted to the rear of the node.
This would allow the temperature sensor to face north instead of south which in
the northern hemisphere would result in much reduced solar warming to the
sensors box as the sun tracks over the southern area of the sky in the UK.

\subsection{Deployment in the field}

\subsection{Placement of nodes}