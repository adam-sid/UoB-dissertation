\section{Overview of hardware}\label{sec:hardware-overview}

The design of the IoT system involved three distinct module types: the sensor
nodes and repeater which are located outside and the gateway which is located
inside a building with WiFi access. The system consists of two separate sensor
nodes that are placed in range of the repeater. The repeater is then placed in
range of the gateway endpoint to allow the uploading of climate data to the
cloud.

\begin{figure}[H]
      \centering
      \includegraphics[width=0.9\textwidth]{contents/part-2/fig2/network-diagram.png}
      \caption{Network diagram of the system}
      \label{fig:network-diagram}
\end{figure} 

\begin{enumerate}
      \item Sensor Node: The two nodes in this part of the network are in the
            perception layer. The nodes collect readings on temperature, humidity,
            wind speed and soil moisture levels. The results are collated into a
            comma separated string which is then emitted as a single packet from
            the LoRa transmitter.
      \item Repeater: This module is part of the network layer of the system as it
            facilitates communication between the perception layer and the
            application layer. The addition of a repeater node effectively doubles
            the range of the system. The repeater reads and decodes received
            LoRa signals from the sensor nodes. It then adds signal strength
            information to the string and re-emits the LoRa signal. The repeater
            has no sensors but is otherwise identical to the sensor nodes.
      \item Gateway: The final part of the hardware system is the gateway - which
            is also part of the network layer. This module has mains power and a
            WiFi connection. The gateway receives LoRa signals from the receiver
            and uploads the data to the cloud.
\end{enumerate}



