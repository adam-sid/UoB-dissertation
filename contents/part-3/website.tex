\section{Backend: Database and SQL API}

For the database, I chose Render as a good low cost option for hosting the web
service as it offers a free trial period, constant uptime and backups. The free
version of web services by Render (render.com), provides a limited set of
resources at zero cost: 256 MB of RAM; 0.1 share of a CPU and 1 GB of storage.
There are also limits on outbound bandwidth, build pipeline minutes and instance
hours. A single instance of a POSTGRES database is available on this service -
which is free for the first 30 days and then costs \$5 a month. These resources
should be sufficient for this project as only small amounts of data are
involved.

The web service is running an API I have written, which will receive data from
the raspberry pi used as the gateway. The data is sent using a POST request with
a JSON body which is parsed, processed and added to the database by the API.
Data can be retrieved for display by the frontend using a GET request.

The database design is relatively simple  add design here with a single table
holding all the data so the choice of database management system is not crucial.
Having said that, POSTGRES is a well established and reliable relational
database management system that complies with ANSI SQL standards and most
importantly for this project is supported by Render.

The 1 GB storage limitation allows for (x) rows of data with the current
database design.

This solution should scale well. The database design allows for additional sites
to be added with very little configuration. And in the unlikely event that the
volumes exceed the Render limits it is easy to pay more for additional
resources.


\section{Frontend: Agriscanner webapp}

The frontend of my project is written in HTML, CSS and Javascript, which is an
archetypal web development stack.

\subsection{picoCSS}

picoCSS is the library I used for the default styling of my webapp. I liked the
its minimal and low distraction look which I thought was ideal for presenting
data. It also had good mobile to desktop scaling which meant my webapp (at least
the non-chart parts) worked on mobile and desktop with little work needed.

\subsection{Apache echarts}

Apache echarts is a javascript library that I have used to build the charts for
my webapp. Data for charts is fetched from the backend and then loaded into a
'series' object. Apache echarts then renders the data series onto a graph with a
variety of options that can be chosen.


\subsection{Making the app mobile friendly}

I wanted the webapp to be accessible for both desktop and mobile users. What
tends to make this difficult is the fact that scaling on desktop and mobile is
normally very different. Also with mobile the longest axis is vertical while on
desktop it tends to be horizontal - so I needed to ensure elements were reactive
to the screen size of the viewer.

picoCSS comes with much of this as standard with normal HTML elements (selection
boxes, divs, titles, navigation bars etc). However integrating Echarts was
difficult as picoCSS styles would not apply to this. I have written a summary
changes I made to ensure that mobile viewers have a good experience.

\begin{enumerate}
    \item Dynamic tooltip: The webapp can tell if the viewer is a touch screen
          and if so it will make the tool tip hover slightly away from the point
          of touch. While on desktop the user will want to hover over a
          datapoint and see the tool tip appear where they are hovering, on
          mobile the digit used to select may cover important tooltip
          information. By making sure the tooltip hovers slightly away from the
          selection this is no longer a problem.
    \item Dynamic chart and font size: Echarts comes with no standard method of
          resizing chart data depending on the size of the device viewing the
          chart. Therefore I developed a number of functions to improve
          readability on mobile devices by dynamically decreasing chart height
          and font size for mobile screen sizes.
    \item
\end{enumerate}



