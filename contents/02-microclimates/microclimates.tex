\section{Microclimates}

\subsection{Microclimates in agriculture}

A microclimate is generally understood as a set of distinct climatic conditions
within a small, localised area \cite{MetOffice2023}. The maximum
size of a microclimate is debated, but the World Meteorological Organisation
(WMO) regards it as occupying an area of anywhere from less than one metre
across to several hundred metres \cite{wmo2024}.  In practice, microclimates can
occur in spaces such as gardens, valleys, caves, or fields. Even human-made
structures can generate their own microclimates; for example, tall buildings can
create \emph{street valleys} that reduce wind flow and lead to the formation of
localised pockets of warmer air, which can also trap higher concentrations of
pollution from vehicle emissions \cite{yang2023}. Vegetation plays a critical
role in influencing microclimates. The addition of trees to an urban environment
can reduce air temperature by as much as \SI{2.8}{\degreeCelsius}
\cite{lai2019}.

This localised climatic variation, characteristic of microclimates, is therefore
significant in agriculture. The climate that crops are exposed has an enormous
impact on overall agricultural yields. Indeed, farmers have modified the
microclimate of crop fields for millennia, a clear example of this being the use
of fencing to reduce soil erosion and damage to edible plants \cite{cleugh1998}.
Therefore, the relationship between microclimates and agriculture has been the
subject of extensive research - particularly as climate change introduces new
threats to food security.

\subsection{Microclimates in apple orchards}




